\documentclass[12pt, oneside]{article}

\usepackage[letterpaper, scale=0.8, centering]{geometry}
\usepackage{fancyhdr}
\setlength{\parindent}{0em}
\setlength{\parskip}{1em}

\pagestyle{fancy}
\fancyhf{}
\renewcommand{\headrulewidth}{0pt}
\rfoot{{\footnotesize Copyright Daniel Grier / Mia Minnes, 2023, Version \today~(\thepage)}}

\usepackage{titlesec}

\author{CSE105Sp23}

\newcommand{\instructions}{{\bf For all HW assignments:} Weekly homework 
may be done individually or in groups of up to 3 students. 
You may switch HW partners for different HW assignments. 
The lowest HW score will not be included in your overall HW average. 
Please ensure your name(s) and PID(s) are clearly visible on the first page of your homework submission 
and then upload the PDF to Gradescope. If working in a group, submit only one submission per group: 
one partner uploads the submission through their Gradescope account and then adds the other group member(s) 
to the Gradescope submission by selecting their name(s) in the ``Add Group Members" dialog box. 
You will need to re-add your group member(s) every time you resubmit a new version of your assignment.
 Each homework question will be graded either for correctness (including clear and precise explanations and 
 justifications of all answers) or fair effort completeness. You may only collaborate on HW with CSE 105 students 
 in your group; if your group has questions about a HW problem, you may ask in drop-in help hours or post a private 
 post (visible only to the Instructors) on Piazza.

All submitted homework for this class must be typed. 
You can use a word processing editor if you like (Microsoft Word, Open Office, Notepad, Vim, Google Docs, etc.) 
but you might find it useful to take this opportunity to learn LaTeX. 
LaTeX is a markup language used widely in computer science and mathematics. 
The homework assignments are typed using LaTeX and you can use the source files 
as templates for typesetting your solutions.
To generate state diagrams of machines, we recommend using Flap.js
or JFLAP. Photographs of clearly hand-drawn diagrams may also be used. We recommend that you
submit early drafts to Gradescope so that in case of any technical difficulties, at least some of your
work is present. You may update your submission as many times as you'd like up to the deadline.


{\bf Integrity reminders}
\begin{itemize}
\item Problems should be solved together, not divided up between the partners. The homework is
designed to give you practice with the main concepts and techniques of the course, 
while getting to know and learn from your classmates.
\item You may not collaborate on homework with anyone other than your group members.
You may ask questions about the homework in office hours (of the instructor, TAs, and/or tutors) and 
on Piazza (as private notes viewable only to the Instructors).  
You \emph{cannot} use any online resources about the course content other than the class material 
from this quarter -- this is primarily to ensure that we all use consistent notation and
definitions (aligned with the textbook) and also to protect the learning experience you will have when
the `aha' moments of solving the problem authentically happen.
\item Do not share written solutions or partial solutions for homework with 
other students in the class who are not in your group. Doing so would dilute their learning 
experience and detract from their success in the class.
\end{itemize}

}

\newcommand{\gradeCorrect}{({\it Graded for correctness}) }
\newcommand{\gradeCorrectFirst}{\gradeCorrect\footnote{This means your solution 
will be evaluated not only on the correctness of your answers, but on your ability
to present your ideas clearly and logically. You should explain how you 
arrived at your conclusions, using
mathematically sound reasoning. Whether you use formal proof techniques or 
write a more informal argument
for why something is true, your answers should always be well-supported. 
Your goal should be to convince the
reader that your results and methods are sound.} }
\newcommand{\gradeComplete}{({\it Graded for completeness}) }
\newcommand{\gradeCompleteFirst}{\gradeComplete\footnote{This means you will 
get full credit so long as your submission demonstrates honest effort to 
answer the question. You will not be penalized for incorrect answers. 
To demonstrate your honest effort in answering the question, we ask 
that you include your attempt to answer *each* part of the question. 
If you get stuck with your attempt, you can still demonstrate 
your effort by explaining where you got stuck and what 
you did to try to get unstuck.} }

\usepackage{tikz}
\usetikzlibrary{automata,positioning,arrows}

\usepackage{amssymb,amsmath,pifont,amsfonts,comment,enumerate,enumitem}
\usepackage{currfile,xstring,hyperref,tabularx,graphicx,wasysym}
\usepackage[labelformat=empty]{caption}
\usepackage[dvipsnames,table]{xcolor}
\usepackage{multicol,multirow,array,listings,tabularx,lastpage,textcomp,booktabs}

% NOTE(joe): This environment is credit @pnpo (https://tex.stackexchange.com/a/218450)
\lstnewenvironment{algorithm}[1][] %defines the algorithm listing environment
{   
    \lstset{ %this is the stype
        mathescape=true,
        frame=tB,
        numbers=left, 
        numberstyle=\tiny,
        basicstyle=\rmfamily\scriptsize, 
        keywordstyle=\color{black}\bfseries,
        keywords={,procedure, div, for, to, input, output, return, datatype, function, in, if, else, foreach, while, begin, end, }
        numbers=left,
        xleftmargin=.04\textwidth,
        #1
    }
}
{}
\lstnewenvironment{java}[1][]
{   
    \lstset{
        language=java,
        mathescape=true,
        frame=tB,
        numbers=left, 
        numberstyle=\tiny,
        basicstyle=\ttfamily\scriptsize, 
        keywordstyle=\color{black}\bfseries,
        keywords={, int, double, for, return, if, else, while, }
        numbers=left,
        xleftmargin=.04\textwidth,
        #1
    }
}
{}

\newcommand\abs[1]{\lvert~#1~\rvert}
\newcommand{\st}{\mid}

\newcommand{\A}[0]{\texttt{A}}
\newcommand{\C}[0]{\texttt{C}}
\newcommand{\G}[0]{\texttt{G}}
\newcommand{\U}[0]{\texttt{U}}

\newcommand{\cmark}{\ding{51}}
\newcommand{\xmark}{\ding{55}}


\newcommand{\SUBSTRING}{\textsc{Substring}}
\newcommand{\REP}{\textsc{Rep}}
\newcommand{\blank}{\scalebox{1.5}{\textvisiblespace}}


\title{Project - CSE 105 Spring 2022}
\date{Part 2 due 5/19/22 at 5pm (no penalty late submission until 8am next day)}

\begin{document}
\maketitle
\thispagestyle{fancy}

\vspace{-30pt}

 The project component of this class will be an opportunity for you to extend your work on 
 assignments and explore applications of your choosing. 
 
 Why?  To go deeper and explore the material from Theory of Computation and how it relates to 
 other aspects of CS and beyond. 
 
 How?  During emergency remote instruction last academic year, we discovered that video 
 assessment and some open-ended personalized projects help ensure fairness and can be less 
 stressful for students than in-person midterm exams. Asynchronous project submission also 
 gives flexibility and allows more physical distancing. 
 
 {\bf Your videos}: We will delete all the videos we receive from you after assigning final grades for 
 the course, and they will be stored in a university-controlled Google Drive directory only 
 accessible to the course staff during the quarter. Please send an email to the instructor 
 (minnes@eng.ucsd.edu) if you have concerns about the video / screencast components of this 
 project or cannot complete projects in this style for some reason. 
 
 You may produce screencasts with any software you choose. One option is to record yourself 
 with Zoom; a tutorial on how to use Zoom to record a screencast (courtesy of Prof. Joe Politz) is 
 here: \href{https://drive.google.com/open?id=1KROMAQuTCk40zwrEFotlYSJJQdcG_GUU}{Tutorial URL}
 The video that was produced from that recording session in Zoom is here:
 \href{{https://drive.google.com/open?id=1MxJN6CQcXqIbOekDYMxjh7mTt1TyRVMl}}{Video produced in tutorial} .
 
 {\bf What resources can you use? }
 This project must be completed individually, without any help from other people, 
 including the course staff (other than logistics support if you get stuck with screencast). 
 You can use any of this quarter’s CSE 105 offering (notes, readings, class videos, 
 supplementary videos, homework feedback). You may additionally search online to respond to 
 project parts that explicitly ask you to do so, and you must  cite all resources (online or offline) 
 that you consult as part of this search. Link directly to the resource and include the name of the 
 author / video creator and the reason you consulted this reference. The work you submit for the 
 project needs to be your own. 
 
The written portion of the project is expected to be clearly legible, and should preferably be typed.

 \newpage
 \section{Tasks for Project Part 2}


 \subsection{Task 1: Explain a review quiz question (Written)}
	
	\begin{enumerate}
		\item Select one question from one of the review quizzes from 4/15/22 (Friday of Week 3) to 4/29/22  (Friday of Week 5) to revisit. Include the problem description, why you picked this question (e.g. what is interesting about it, what is hard about it, or why you wanted to take a second look at it), and your solution. Question selection: you can pick any one question listed in the Gradescope review quizzes, and you must address  all  of its parts. 
 		\item For each part of your chosen question: prepare a complete solution (you can use the homework solutions we post for guidance about the style). Your submission will be evaluated not only on the correctness of your answers, but on your ability to present your ideas clearly and logically. You should explain how you arrived at your conclusions, using mathematically sound reasoning. Your goal should be to convince the reader that your results and methods are sound. Imagine you are preparing these solutions for someone else taking CSE 105 who missed that week and is “catching up”
 
 		\item Include at least 2 potential mistakes that a student may have made while attempting to solve the quiz problem that you selected. Explain why the reasoning behind these mistakes is flawed so that a student reading this may learn from these mistakes. It’s a good idea to include mistakes that you made when you first tried to solve this problem!	
	\end{enumerate}
	
	{\bf Style guidelines}: your written submission for Task 1 should clearly label the three parts:
	{\it Question Selection}, {\it Solution},  and {\it Potential Mistakes}.

\newpage
\subsection{Task 2: Closure of the Collection of Regular Languages}
	
	For this task, we fix $\Sigma = \{0,1\}$. Recall that the composition of two 
	functions $f$ and $g$ is denoted $f \circ g$, which can also be written as $f(g(x))$, and is the 
	result of first applying the function $g$ to the input $x$ (producing $g(x)$), and then applying $f$ to $g(x)$. 
	Below are some functions with domain and codomain $\mathcal{P}(\Sigma^*)$; that is, they 
	each take in a language over $\Sigma$
	and output a language over $\Sigma$. 
	
	\begin{align*}
		TZ(L) &= \{ w0^k \mid w \in L, k \geq 0 \}\\
		R(L) &= \{ w \mid w^R \in L\} \textrm{(where $w^R$ is reversing $w$, e.g. $(100)^R = 001$)}\\
		E(L) &= \{ w \mid w \in L, \textrm{ the length of $w$ is even} \} \\
		T(L) &= \{ w \mid w \in L, \textrm{ the length of $w$ is a multiple of } 3 \}\\
		EQ(L) &= \{ w^k1^k \mid k \geq 0, w \in L \} \textrm{(where $w^k$ is concatenating $w$ with itself $k$ times, e.g. $(100)^2 = 100100$)}\\
		LT(L) &= \{ w^k1^j \mid 0 \leq k < j , w \in L \}\\
		EQ2(L) &= \{ w^kx^k \mid k \geq 0, w \in L, x \in L \}
	\end{align*}

	For example $(TZ \circ R)(L) = TZ(R(L)) = \{ w^R 0^k | w \in L, k \geq 0\}$.

	
	\begin{enumerate}
		\item Choose two functions from the above list so that their 
		composition, $h$, is such that the collection of regular languages over $\Sigma$ is closed under $h$.
		
		\begin{enumerate}
			\item Provide a clear and complete definition of  your function $h$.
			A complete solution will clearly specify the two functions you chose to compose, 
			the order in which $h$ applies them, and a general description of what the function $h$ does using set builder
			descriptions and/or English prose.
			{\bf Note:} You do not need to apply $h$ to a language in this step, you only need to define $h$.
			\item Prove that the collection of regular languages over $\Sigma$ is 
			closed under $h$ by writing out the following argument in detail: 
			Consider an arbitrary regular language $L$ over the alphabet $\Sigma = \{0, 1\}$.
			Since it is regular, it is recognized by a DFA and let $M = (Q, \Sigma, \delta, q_0, F)$ 
			be such a DFA over $\Sigma$ with $L(M) = L$. 
			Give the formal construction of an NFA $N$ with $L(N) = h(L)$ for your function $h$.
			Briefly justify this construction by tracing the computations of $N$ and/or referencing constructions
			we discussed in class and in the book. In particular, explain the role of each parameter in the definition of $N$ in 
			the construction.
		\end{enumerate}
		
		\item Choose two functions from the above list so that their composition, $h'$, 
		is such that the collection of regular languages over $\Sigma$ is {\bf not} closed under $h'$. 
		{\it Note the functions you choose for this part may or may not overlap with those from the previous part; it's up to you to decide.}
		
		\begin{enumerate}
			\item Provide a clear and complete definition of  your function $h'$.
			A complete solution will clearly specify the two functions you chose to compose, 
			the order in which $h'$ applies them, and a general description of what 
			the function $h'$ does using set builder
			descriptions and/or English prose.			
			\item Give a witness language $L$ that can be used to prove that the class of regular
			languages over $\Sigma$ is not closed under $h'$. To do so: 
			(1) clearly define a language $L$ over $\Sigma$, (2) prove that $L$ is regular, and (3) prove that $h'(L)$ is not regular.
			You may use results proved in class and / or the relevant sections in the textbook as part of your proofs 
			if you would like, but you must label these results and provide references to the day we discussed them and/or the 
			page number in the book.
		\end{enumerate}
	\end{enumerate}
		
\subsection{Task 3: Implementation examples and Video}
 With the introduction of PDAs, our models of computation begin to approach the power that 
 modern day computers have. 
 Choose a specific non-regular but context-free language mentioned in some question 
 in the review quizzes between 4/15/22 (Friday of Week 3) 
 to 4/29/22 (Friday of Week 5); you will write a program in Java, Python, JavaScript, or C++
 which is able to test membership of strings in 
 that language.
 The program you write should function like a PDA, using a constant amount of memory plus 
 access to a stack, 
 and should only make a single pass through the string.
	

 Presenting your reasoning and demonstrating it via screenshare are important skills that 
 also show us a lot of your learning. Getting practice with this style of presentation is a 
 good thing for you to learn in general and a rich way for us to assess your skills. Create 
 a 3-5 minute screencast video with the following components:
 \begin{itemize}
	\item Start with your face and your student ID for a few seconds at the beginning, and introduce yourself audibly while on screen. 
	You don't have to be on camera for the rest of the video, though it's fine if you are. 
	We are looking for a brief confirmation that it's you creating the video and 
	doing the work you submitted.
	\item State which language you chose from the review quiz, and show the state diagram for 
	a PDA which recognizes the language, briefly justifying why it works.
	\item State which programming language you chose to use and show on the 
	screen all the code your wrote to implement the PDA in your chosen programming language.
	Discuss how the behavior of your program is related to the state diagram of the PDA, 
	and discuss the implementation choices you made when creating this program.
	\item Demonstrate 4 test cases (2 strings in the language recognized by your PDA, 
	2 strings not in this language), clearly defining each one, 
	explaining the expected behavior of the PDA, and 
	showing the output / feedback your program gives to indicate whether the expected behavior 
	matches the actual behavior.
\end{itemize}
You will submit this video along with a written version of Tasks 1 and 2 to Gradescope.

{\bf Extra exploration (not for credit)}: What would it take to implement context-free grammars in code? 
Could you use any of your work from implementaing PDAs?

	
\section{Grading Criteria and checklists}

{\bf Task 1}

Submission covers a complete review quiz question from the relevant weeks 
(all parts of the question must be addressed for multi-part questions).

Submission clearly labels review questions, including which day it's from and the problem description.

Submission includes why the student picked the problem/ what they found interesting.

Solution is written (or typed) out in detail step-by-step, with clear and correct logic and justification.

Submission includes 2 potential mistakes that a student might make while solving this question 
and explains why they are wrong.


{\bf Task 2}

{\it Question 1}: 

The function $h$ is described clearly and completely, using appropriate notation and terminology.

The formal construction of $N$ is clear, correct, and complete, and is justified appropriately
and correctly.

{\it Question 2}:

The function $h'$ is described clearly and completely, using appropriate notation and terminology.
    
The language $L$ is specified clearly and completely and is a viable witness 
for the proof.

The proof that $L$ is regular is clear, correct, and complete.

The proof that $h'(L)$ is not regular is clear, correct, and complete.

{\bf Task 3}

Logistics Items
\begin{itemize}
    \item Video loads correctly
    \item Video is between 3 and 5 minutes
    \item Video shows the student's face and ID, and they 
	introduce themself audibly while on screen
\end{itemize}

The video clearly states which language was chosen for study, 
and references a specific review quiz with this language.

The video shows the state diagram of a PDA which recognizes the 
chosen language.

The video clearly describes which programming language was chosen 
for the implementaiton and gives the reasons why.

The video discusses the connections between the state diagram of the PDA 
and its implementation in the code.

The video clearly demonstrates all test cases, including both expected
and actual output. The video should include screencasts of 
running the code live to demonstrate these test cases.

\end{document}