\documentclass[12pt, oneside]{article}

\usepackage[letterpaper, scale=0.8, centering]{geometry}
\usepackage{fancyhdr}
\setlength{\parindent}{0em}
\setlength{\parskip}{1em}

\pagestyle{fancy}
\fancyhf{}
\renewcommand{\headrulewidth}{0pt}
\rfoot{{\footnotesize Copyright Daniel Grier / Mia Minnes, 2023, Version \today~(\thepage)}}

\usepackage{titlesec}

\author{CSE105Sp23}

\newcommand{\instructions}{{\bf For all HW assignments:} Weekly homework 
may be done individually or in groups of up to 3 students. 
You may switch HW partners for different HW assignments. 
The lowest HW score will not be included in your overall HW average. 
Please ensure your name(s) and PID(s) are clearly visible on the first page of your homework submission 
and then upload the PDF to Gradescope. If working in a group, submit only one submission per group: 
one partner uploads the submission through their Gradescope account and then adds the other group member(s) 
to the Gradescope submission by selecting their name(s) in the ``Add Group Members" dialog box. 
You will need to re-add your group member(s) every time you resubmit a new version of your assignment.
 Each homework question will be graded either for correctness (including clear and precise explanations and 
 justifications of all answers) or fair effort completeness. You may only collaborate on HW with CSE 105 students 
 in your group; if your group has questions about a HW problem, you may ask in drop-in help hours or post a private 
 post (visible only to the Instructors) on Piazza.

All submitted homework for this class must be typed. 
You can use a word processing editor if you like (Microsoft Word, Open Office, Notepad, Vim, Google Docs, etc.) 
but you might find it useful to take this opportunity to learn LaTeX. 
LaTeX is a markup language used widely in computer science and mathematics. 
The homework assignments are typed using LaTeX and you can use the source files 
as templates for typesetting your solutions.
To generate state diagrams of machines, we recommend using Flap.js
or JFLAP. Photographs of clearly hand-drawn diagrams may also be used. We recommend that you
submit early drafts to Gradescope so that in case of any technical difficulties, at least some of your
work is present. You may update your submission as many times as you'd like up to the deadline.


{\bf Integrity reminders}
\begin{itemize}
\item Problems should be solved together, not divided up between the partners. The homework is
designed to give you practice with the main concepts and techniques of the course, 
while getting to know and learn from your classmates.
\item You may not collaborate on homework with anyone other than your group members.
You may ask questions about the homework in office hours (of the instructor, TAs, and/or tutors) and 
on Piazza (as private notes viewable only to the Instructors).  
You \emph{cannot} use any online resources about the course content other than the class material 
from this quarter -- this is primarily to ensure that we all use consistent notation and
definitions (aligned with the textbook) and also to protect the learning experience you will have when
the `aha' moments of solving the problem authentically happen.
\item Do not share written solutions or partial solutions for homework with 
other students in the class who are not in your group. Doing so would dilute their learning 
experience and detract from their success in the class.
\end{itemize}

}

\newcommand{\gradeCorrect}{({\it Graded for correctness}) }
\newcommand{\gradeCorrectFirst}{\gradeCorrect\footnote{This means your solution 
will be evaluated not only on the correctness of your answers, but on your ability
to present your ideas clearly and logically. You should explain how you 
arrived at your conclusions, using
mathematically sound reasoning. Whether you use formal proof techniques or 
write a more informal argument
for why something is true, your answers should always be well-supported. 
Your goal should be to convince the
reader that your results and methods are sound.} }
\newcommand{\gradeComplete}{({\it Graded for completeness}) }
\newcommand{\gradeCompleteFirst}{\gradeComplete\footnote{This means you will 
get full credit so long as your submission demonstrates honest effort to 
answer the question. You will not be penalized for incorrect answers. 
To demonstrate your honest effort in answering the question, we ask 
that you include your attempt to answer *each* part of the question. 
If you get stuck with your attempt, you can still demonstrate 
your effort by explaining where you got stuck and what 
you did to try to get unstuck.} }

\usepackage{tikz}
\usetikzlibrary{automata,positioning,arrows}

\usepackage{amssymb,amsmath,pifont,amsfonts,comment,enumerate,enumitem}
\usepackage{currfile,xstring,hyperref,tabularx,graphicx,wasysym}
\usepackage[labelformat=empty]{caption}
\usepackage[dvipsnames,table]{xcolor}
\usepackage{multicol,multirow,array,listings,tabularx,lastpage,textcomp,booktabs}

% NOTE(joe): This environment is credit @pnpo (https://tex.stackexchange.com/a/218450)
\lstnewenvironment{algorithm}[1][] %defines the algorithm listing environment
{   
    \lstset{ %this is the stype
        mathescape=true,
        frame=tB,
        numbers=left, 
        numberstyle=\tiny,
        basicstyle=\rmfamily\scriptsize, 
        keywordstyle=\color{black}\bfseries,
        keywords={,procedure, div, for, to, input, output, return, datatype, function, in, if, else, foreach, while, begin, end, }
        numbers=left,
        xleftmargin=.04\textwidth,
        #1
    }
}
{}
\lstnewenvironment{java}[1][]
{   
    \lstset{
        language=java,
        mathescape=true,
        frame=tB,
        numbers=left, 
        numberstyle=\tiny,
        basicstyle=\ttfamily\scriptsize, 
        keywordstyle=\color{black}\bfseries,
        keywords={, int, double, for, return, if, else, while, }
        numbers=left,
        xleftmargin=.04\textwidth,
        #1
    }
}
{}

\newcommand\abs[1]{\lvert~#1~\rvert}
\newcommand{\st}{\mid}

\newcommand{\A}[0]{\texttt{A}}
\newcommand{\C}[0]{\texttt{C}}
\newcommand{\G}[0]{\texttt{G}}
\newcommand{\U}[0]{\texttt{U}}

\newcommand{\cmark}{\ding{51}}
\newcommand{\xmark}{\ding{55}}


\newcommand{\SUBSTRING}{\textsc{Substring}}
\newcommand{\REP}{\textsc{Rep}}
\newcommand{\blank}{\scalebox{1.5}{\textvisiblespace}}

\titleformat{\subsubsection}[runin]
   {\normalfont\bfseries}{}{}{}
   
\title{Project - CSE 105 Spring 2022}
\date{Due 6/1/23 at 5pm (no penalty late submission until 8am next day)}

\begin{document}
\maketitle

\thispagestyle{fancy}

The project component is designed for you to go deeper and extend your work on assignments 
and to explore an application of your choosing. 
The project is an individual assignment and has 
two tasks: 
\begin{itemize}
    \item Task 1: A meaningful language (written) and \item Task 2: A helpful function (some programming, presented as a screencast video).
\end{itemize}
\vspace{-20pt}


\subsubsection*{What resources can you use?}
This project must be completed individually, without any help from other people, 
including the course staff (other than logistics support if you get stuck with screencast). 

You can use any of this quarter's CSE 105 offering (notes, readings, class videos, homework feedback). 
These resources should be more than enough. 

If you are struggling to get started and want to 
look elsewhere online, you must acknowledge this by listing and 
citing any resources you consult 
(even if you do not explicitly quote them), including any large-language model style resources. 
Link directly to them and include the name of the 
author / video creator, any search strings or prompts you used, and the reason you consulted this reference. 

The work you submit for 
the project needs to be your own. Again, you shouldn't need to look anywhere other 
than this quarter's material and doing so may result in definitions or notations 
that conflict with our norms in this class so think carefully before you go down this path.

\subsubsection*{Submitting the project}
You will submit a PDF for Task 1 and a video for Task 2 
to an online assignment on Gradescope.

\newpage
\subsection*{Task 1: A meaningful language}

Automaton models are useful for concisely representing patterns. 
In this question, you'll choose a pattern in an application you care about, 
define it precisely, and then build a DFA, NFA, or PDA that 
recognizes it.

First, pick {\bf one} application for your example. 
Here are some ideas to 
get you started - but you can choose to go in a different direction all together.
\begin{itemize}
\item Data validation for input in text files (e.g. emails with 
specific domains, dates in specific formats, PIDs in a class list, etc.)
\item Finding ASCII codes for punctuation in a binary file.
\item The CDC recommended procedure for hand washing (Refer to the guidelines from the CDC here https://www.cdc.gov/handwashing/index.html in your explanation. You might find the first example in chapter 1 about automatic door controllers helpful when starting your design.)
\item See more ideas here: 

{\scriptsize \url{https://theory-cs.github.io/files/practical-applications-of-theory-of-computation.pdf}}
\end{itemize}

Then:
\begin{enumerate}
    \item In a paragraph or so, give the context for your chosen
    application and why you chose it.
    \item Specify the alphabet for your example.
    \item Write a precise (mathematical and/or English) description
    of a set of strings over this alphabet that is important, 
    and include a sentence or so justifying why this set is important.
    \item Give one example of a string in this set and a string not 
    in this set, and explain why you chose these example strings.
    \item Classify your language as regular or context-free, and prove
    this classification by giving an appropriate machine that
    recognizes your language. Justify your construction.
\end{enumerate}

\subsubsection*{Grading criteria and checklists}

Solution is typed out in detail step-by-step, with clear and 
correct logic and justification.

Each of the five items are included, with precise language and notation
for all terms and complete, correct, and clear justification.

\newpage
\subsection*{Task 2: A helpful function}

To relate the difficulty level of one language to another 
we use mapping reduction, which relies
on the notion of computable function. In this part of the project, 
you will define, program, and trace a specific 
computable function from $\{a,b\}^*$ to $\{a,b\}^*$.

First, choose a function you will be working with. You can pick any function you like so long as:
	\begin{itemize}
		\item Its domain is $\{a,b\}^*$ and its 
                codomain is $\{a,b\}^*$
		\item There is at least one domain 
                element that is mapped to a string that is 
                {\bf longer} than it and there is at least one 
                domain element that is mapped to a string that is 
                {\bf shorter} than it.
	\end{itemize}
Then:
 \begin{enumerate}
    \item Give a high-level description of a Turing machine that computes
    this function.
    \item Draw the state diagram of this Turing machine.
    \item Write a program in Java, Python, JavaScript, or C++ (or 
        another programming language of your choosing)
        that simulates the computations of this Turing machine.
        Your program should display a snapshot of each step of the computation, 
        including the state of the machine, the current (non-blank)
        contents of the tape, and the location of the read/write head
        of the Turing machine. If you would like, you may 
        use aids such as co-pilot or ChatGPT to help you write this 
        program. However, you should test the code that is 
        produced and be able 
        to explain what it is doing. As a header in 
        your code file, include a comment block describing 
        any resources that were used to help
        generate your code.
    \item Run your program on a string over $\{a,b\}$ which 
        the function maps to a string that is longer than it, and
    \item Run your program on a string over $\{a,b\}$ which 
        the function maps to a string that is shorter than it.
\end{enumerate}

 Presenting your reasoning and demonstrating it via screenshare are important skills that 
 also show us a lot of your learning. Getting practice with this style of presentation is a 
 good thing for you to learn in general and a rich way for us to assess your skills. To demonstrate your work, you will create 
 a 3-5 minute screencast video with the following components:
 \begin{itemize}
	\item Start with your face and your student ID for a 
        few seconds at the beginning, and introduce yourself audibly 
        while on screen. 
	You don't have to be on camera for the rest of the 
        video, though it's fine if you are. 
	We are looking for a brief confirmation that it's 
        you creating the video and 
	doing the work you submitted.
	\item Present the function you will be working with.
	Your video should include a clear and precise definition of the               function (before you introduce any Turing machines).
	\item Show on the screen and explain the high-level description of
        your Turing machine witnessing that your function is computable.
	\item Show on the screen and explain the state diagram of 
        your Turing machine.
	\item Show on the screen and present your code, including
        the software design choices you made (e.g.\ which data structures
        are you using, etc.) and any resources you used.
        \item Demonstrate running your code on the two 
        example input strings specified above. Explain why the output 
        of your program is what you would expect.
\end{itemize}


\subsubsection*{Grading criteria and checklists}

Logistics: video loads correctly, is between 3 and 5 minutes, 
shows the student's face and ID, and they introduce themself 
audibly while on screen.

The video clearly states which function was chosen for study, 
the function which is well-defined and computable, the video
presents the two different descriptions of the Turing machine clearly,
and the Turing machine correctly computes the function.

The video clearly describes which programming language was chosen 
for the implementation and gives the reasons why.

The video discusses the connections between the state diagram of the 
Turing machine
and its implementation in the code.

The video clearly demonstrates all test cases, including both expected
and actual output. The video should include screencasts of 
running the code live to demonstrate these test cases.


\subsubsection*{Your video:} You may produce screencasts 
with any software you choose. 
One option is to record yourself with Zoom; a tutorial on how to use 
Zoom to record a 
screencast (courtesy of Prof. Joe Politz)  is here: 

\url{https://drive.google.com/open?id=1KROMAQuTCk40zwrEFotlYSJJQdcG_GUU}.

The video that was produced from that recording session in Zoom is here:

\url{https://drive.google.com/open?id=1MxJN6CQcXqIbOekDYMxjh7mTt1TyRVMl}

Please send an email to the instructors 
(minnes@ucsd.edu and 
dgrier@ucsd.edu) if you have 
concerns about  the video / screencast components of this project or 
cannot complete projects in this style for some reason.

\end{document}