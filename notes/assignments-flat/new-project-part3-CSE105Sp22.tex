\documentclass[12pt, oneside]{article}

\usepackage[letterpaper, scale=0.8, centering]{geometry}
\usepackage{fancyhdr}
\setlength{\parindent}{0em}
\setlength{\parskip}{1em}

\pagestyle{fancy}
\fancyhf{}
\renewcommand{\headrulewidth}{0pt}
\rfoot{{\footnotesize Copyright Mia Minnes, 2022, Version \today~(\thepage)}}

\author{CSE105Sp22}

\newcommand{\instructions}{{\bf For all HW assignments:}

Weekly homework may be done individually or in groups of up to 3 students. 
You may switch HW partners for different HW assignments. 
The lowest HW score will not be included in your overall HW average. 
Please ensure your name(s) and PID(s) are clearly visible on the first page of your homework submission 
and then upload the PDF to Gradescope. If working in a group, submit only one submission per group: 
one partner uploads the submission through their Gradescope account and then adds the other group member(s) 
to the Gradescope submission by selecting their name(s) in the ``Add Group Members" dialog box. 
You will need to re-add your group member(s) every time you resubmit a new version of your assignment.
 Each homework question will be graded either for correctness (including clear and precise explanations and 
 justifications of all answers) or fair effort completeness. You may only collaborate on HW with CSE 105 students 
 in your group; if your group has questions about a HW problem, you may ask in drop-in help hours or post a private 
 post (visible only to the Instructors) on Piazza.

All submitted homework for this class must be typed. 
You can use a word processing editor if you like (Microsoft Word, Open Office, Notepad, Vim, Google Docs, etc.) 
but you might find it useful to take this opportunity to learn LaTeX. 
LaTeX is a markup language used widely in computer science and mathematics. 
The homework assignments are typed using LaTeX and you can use the source files 
as templates for typesetting your solutions.
To generate state diagrams of machines, we recommend using Flap.js
or JFLAP. Photographs of clearly hand-drawn diagrams may also be used. We recommend that you
submit early drafts to Gradescope so that in case of any technical difficulties, at least some of your
work is present. You may update your submission as many times as you'd like up to the deadline.


{\bf Integrity reminders}
\begin{itemize}
\item Problems should be solved together, not divided up between the partners. The homework is
designed to give you practice with the main concepts and techniques of the course, 
while getting to know and learn from your classmates.
\item You may not collaborate on homework with anyone other than your group members.
You may ask questions about the homework in office hours (of the instructor, TAs, and/or tutors) and 
on Piazza (as private notes viewable only to the Instructors).  
You \emph{cannot} use any online resources about the course content other than the class material 
from this quarter -- this is primarily to ensure that we all use consistent notation and
definitions we will use this quarter and also to protect the learning experience you will have when
the `aha' moments of solving the problem authentically happen.
\item Do not share written solutions or partial solutions for homework with 
other students in the class who are not in your group. Doing so would dilute their learning 
experience and detract from their success in the class.
\end{itemize}

}
\usepackage{amssymb,amsmath,pifont,amsfonts,comment,enumerate,enumitem}
\usepackage{currfile,xstring,hyperref,tabularx,graphicx,wasysym}
\usepackage[labelformat=empty]{caption}
\usepackage[dvipsnames,table]{xcolor}
\usepackage{multicol,multirow,array,listings,tabularx,lastpage,textcomp,booktabs}

\lstnewenvironment{algorithm}[1][] {   
    \lstset{ mathescape=true,
        frame=tB,
        numbers=left, 
        numberstyle=\tiny,
        basicstyle=\rmfamily\scriptsize, 
        keywordstyle=\color{black}\bfseries,
        keywords={,procedure, div, for, to, input, output, return, datatype, function, in, if, else, foreach, while, begin, end, }
        numbers=left,
        xleftmargin=.04\textwidth,
        #1
    }
}
{}
\lstnewenvironment{java}[1][]
{   
    \lstset{
        language=java,
        mathescape=true,
        frame=tB,
        numbers=left, 
        numberstyle=\tiny,
        basicstyle=\ttfamily\scriptsize, 
        keywordstyle=\color{black}\bfseries,
        keywords={, int, double, for, return, if, else, while, }
        numbers=left,
        xleftmargin=.04\textwidth,
        #1
    }
}
{}

\newcommand\abs[1]{\lvert~#1~\rvert}
\newcommand{\st}{\mid}

\newcommand{\A}[0]{\texttt{A}}
\newcommand{\C}[0]{\texttt{C}}
\newcommand{\G}[0]{\texttt{G}}
\newcommand{\U}[0]{\texttt{U}}

\newcommand{\cmark}{\ding{51}}
\newcommand{\xmark}{\ding{55}}
 
 
\title{Project - CSE 105 Spring 2022}
\date{Part 3 due 6/2/22 at 5pm (no penalty late submission until 8am next day)}

\begin{document}
\maketitle
\thispagestyle{fancy}

\vspace{-30pt}

 The project component of this class will be an opportunity for you to extend your work on 
 assignments and explore applications of your choosing. 
 
 Why?  To go deeper and explore the material from Theory of Computation and how it relates to 
 other aspects of CS and beyond. 
 
 How?  During emergency remote instruction last academic year, we discovered that video 
 assessment and some open-ended personalized projects help ensure fairness and can be less 
 stressful for students than in-person midterm exams. Asynchronous project submission also 
 gives flexibility and allows more physical distancing. 
 
 {\bf Your videos}: We will delete all the videos we receive from you after assigning final grades for 
 the course, and they will be stored in a university-controlled Google Drive directory only 
 accessible to the course staff during the quarter. Please send an email to the instructor 
 (minnes@eng.ucsd.edu) if you have concerns about the video / screencast components of this 
 project or cannot complete projects in this style for some reason. 
 
 You may produce screencasts with any software you choose. One option is to record yourself 
 with Zoom; a tutorial on how to use Zoom to record a screencast (courtesy of Prof. Joe Politz) is 
 here: \href{https://drive.google.com/open?id=1KROMAQuTCk40zwrEFotlYSJJQdcG_GUU}{Tutorial URL}
 The video that was produced from that recording session in Zoom is here:
 \href{{https://drive.google.com/open?id=1MxJN6CQcXqIbOekDYMxjh7mTt1TyRVMl}}{Video produced in tutorial} .
 
 {\bf What resources can you use? }
 This project must be completed individually, without any help from other people, 
 including the course staff (other than logistics support if you get stuck with screencast). 
 You can use any of this quarter’s CSE 105 offering (notes, readings, class videos, 
 supplementary videos, homework feedback). You may additionally search online to respond to 
 project parts that explicitly ask you to do so, and you must  cite all resources (online or offline) 
 that you consult as part of this search. Link directly to the resource and include the name of the 
 author / video creator and the reason you consulted this reference. The work you submit for the 
 project needs to be your own. 
 
The written portion of the project is expected to be clearly legible, and should preferably be typed.

 \newpage
 \section*{Tasks for Project Part 3}

 \subsection*{Task 1: Explain a review quiz question (Written)}
	
	\begin{enumerate}
		\item[(a)] Select one question from one of the review quizzes from 5/2/22 (Monday of Week 6) to 5/27/22  (Friday of Week 9) to revisit.
		Include the problem description, why you picked this question (e.g. what is interesting about it, what is hard about it, 
		or why you wanted to take a second look at it), and your solution. Question selection: 
		you can pick any one question listed in the Gradescope review quizzes, and you must address 
		all  of its parts. 
 		\item[(b)] For each part of your chosen question: prepare a complete solution 
		 (you can use the homework solutions we post for guidance about the style). 
		 Your submission will be evaluated not only on the correctness of your answers, 
		 but on your ability to present your ideas clearly and logically. You should explain how you arrived at your conclusions, 
		 using mathematically sound reasoning. Your goal should be to convince the reader that your results and 
		 methods are sound. Imagine you are preparing these solutions for someone else taking 
		 CSE 105 who missed that week and is “catching up”.
 
 		\item[(c)] Include at least 2 potential mistakes that a student may have made while attempting to solve the quiz 
		 problem that you selected. Explain why the reasoning behind these mistakes is flawed so that 
		 a student reading this may learn from these mistakes. It's a good idea to include mistakes that you made 
		 when you first tried to solve this problem!	
	\end{enumerate}
	
	{\bf Style guidelines}: your written submission for Task 1 should clearly label the three parts:
	{\it Question Selection}, {\it Solution},  and {\it Potential Mistakes}.

\subsection*{Task 2: Proving undecidability with mapping reductions (Written)}

	Define:
	\begin{align*}
		L_n &= \{ \langle M \rangle \mid \textrm{$M$ is a Turing machine and } |L(M)| = n\} \\
		X_{y} & = \{ \langle M \rangle \mid \textrm{$M$ is a Turing machine and } y \in L(M)\} \\
	\end{align*}

	Option 1: Pick a specific positive integer $n$ and you will show that $L_n$ is undecidable.

	Option 2: Pick a specific string $y$ over $\{0,1\}$ and you will show that $X_y$ is undecidable.

	For either option:
	\begin{enumerate}
		\item[(a)] Clearly specify whether you chose Option 1 or Option 2, and specify the value of $n$ or $y$ you picked.
		\item[(b)] Give two specific examples of strings in the set $L_n$ or $X_y$, and two specific examples 
		of strings not in the set. Justify your examples with specific connections between the strings and the 
		definition of the set.
		\item[(c)] Pick whether you will mapping reduce $A_{TM}$, $HALT_{TM}$, $\overline{A_{TM}}$, or $\overline{HALT_{TM}}$
		to your set. Define {\bf two different} computable functions that can witness the mapping reduction.
		Prove that each of these functions witnesses the mapping reduction.
	\end{enumerate}

	If you get stuck:

	We want you to demonstrate your knowledge about mapping reductions in this part of the project.  
	As professionals, it's important to realize when we don't know or unsure about something.
	In grading your work on this part of the project, some partial credit will be available for 
	partial correct progress on the task and then explanations of where
	you got stuck and what you did to try to get unstuck.

		
\subsection*{Task 3: Video about computable functions}
To relate the difficulty level of one language to another we use mapping reduction, which relies
on the notion of computable function. In this part of the project, you will define and explain a specific 
computable function from $\{0,1\}^*$ to $\{0,1\}^*$.

 Presenting your reasoning and demonstrating it via screenshare are important skills that 
 also show us a lot of your learning. Getting practice with this style of presentation is a 
 good thing for you to learn in general and a rich way for us to assess your skills. Create 
 a 3-5 minute screencast video with the following components:
 \begin{itemize}
	\item Start with your face and your student ID for a few seconds at the beginning, and introduce yourself audibly while on screen. 
	You don't have to be on camera for the rest of the video, though it's fine if you are. 
	We are looking for a brief confirmation that it's you creating the video and 
	doing the work you submitted.
	\item Present the function you will be working with. You can pick any function you like so long as:
	\begin{itemize}
		\item Its domain is $\{0,1\}^*$ and its codomain is $\{0,1\}^*$
		\item It is not the identity map (that sends every string to itself), and it is not the function we 
		worked through in class $f_1: \{0,1\}^* \to \{0,1\}^*$ where $f_1(x) = x0$.
	\end{itemize}
	Your video should include a clear and precise definition of the function.
	\item Give a high-level description of a Turing machine witnessing that your function is computable.
	\item Present the state diagram and formal definition of a Turing machine witnessing that your function is computable.
	\item Trace the computation of the Turing machine whose state diagram you gave on an input of length $3$.
\end{itemize}
You will submit this video along with a written version of Tasks 1 and 2 to Gradescope.

{\bf Extra exploration (not for credit)}: What would it take to implement your computable function in code in a programming language
of your choosing? Could you use this computable function to witness any mapping reductions?

	
\section{Grading Criteria and checklists}

{\bf Task 1}

Submission covers a complete review quiz question from the relevant weeks 
(all parts of the question must be addressed for multi-part questions).

Submission clearly labels review questions, including which day it's from and the problem description.

Submission includes why the student picked the problem/ what they found interesting.

Solution is written (or typed) out in detail step-by-step, with clear and correct logic and justification.

Submission includes 2 potential mistakes that a student might make while solving this question 
and explains why they are wrong.


{\bf Task 2}

Submission clearly specify whether Option 1 or Option 2 is chosen, and clearly specifies the value of $n$ or $y$ as appropriate.

Two specific examples of strings in the set and two specific examples 
of strings not in the set are included. Justifications of membership / non-membership are complete, clear, correct, and precise.
Explanations include specific refereence to the example and to relevant definitions.

Each of the two mapping reductions clearly identify the sets involved and include a high-level definition for
a Turing machine witnessing the mapping reduction, an analysis of the output of the function for possible inputs, and
a connection with the definition of mapping reduction. Definitions and explanations are complete, clear, correct, and precise.	


{\bf Task 3}

Logistics Items
\begin{itemize}
    \item Video loads correctly
    \item Video is between 3 and 5 minutes
    \item Video shows the student's face and ID, and they 
	introduce themselves audibly while on screen.
\end{itemize}

The video clearly presents a function which is well-defined and computable.

The video presents a correct high-level description of a Turing machine that computes this function.

The video presents a complete and correct formal definition of a Turing machine that computes this function,
including a state diagram.

The video includes a trace of the computation of this Turing machine on an input of length $3$, where
each step of the trace is included and shows the contents of the tape, the location of the read-write head, and 
the control state of the machine. The trace compares the Turing machine behavior with the expected output
of the function on this input string.

\end{document}