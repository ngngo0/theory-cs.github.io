\documentclass[12pt, oneside]{article}

\usepackage[letterpaper, scale=0.89, centering]{geometry}
\usepackage{fancyhdr}
\setlength{\parindent}{0em}
\setlength{\parskip}{1em}

\pagestyle{fancy}
\fancyhf{}
\renewcommand{\headrulewidth}{0pt}
\rfoot{\href{https://creativecommons.org/licenses/by-nc-sa/2.0/}{CC BY-NC-SA 2.0} Version \today~(\thepage)}

\usepackage{amssymb,amsmath,pifont,amsfonts,comment,enumerate,enumitem}
\usepackage{currfile,xstring,hyperref,tabularx,graphicx,wasysym}
\usepackage[labelformat=empty]{caption}
\usepackage{xcolor}
\usepackage{multicol,multirow,array,listings,tabularx,lastpage,textcomp,booktabs}

\lstnewenvironment{algorithm}[1][] {   
    \lstset{ mathescape=true,
        frame=tB,
        numbers=left, 
        numberstyle=\tiny,
        basicstyle=\rmfamily\scriptsize, 
        keywordstyle=\color{black}\bfseries,
        keywords={,procedure, div, for, to, input, output, return, datatype, function, in, if, else, foreach, while, begin, end, }
        numbers=left,
        xleftmargin=.04\textwidth,
        #1
    }
}
{}

\newcommand\abs[1]{\lvert~#1~\rvert}
\newcommand{\st}{\mid}

\newcommand{\cmark}{\ding{51}}
\newcommand{\xmark}{\ding{55}}
 
\begin{document}
\begin{flushright}
    \StrBefore{\currfilename}{.}
\end{flushright} \section*{Week1 monday}


We will use vocabulary that should be familiar from your discrete 
math and introduction to proofs classes.  Some of the notation conventions may 
be a bit different: we will use the notation from this class' textbook\footnote{Page references are to 
the 3rd edition (International) of Sipser's Introduction to the Theory of Computation,
available through various sources for under \$30. You may be able to 
opt in to purchase a digital copy through Canvas. Copies of the book are also available 
for those who can't access the book
to borrow from the course instructor, while supplies last (minnes@eng.ucsd.edu)}.

Write out in words the meaning of the symbols below: 
\[
    \{ a,b, c\}
\]

\phantom{The set whose elements are $a$, $b$, and $c$}

\[
    | \{a, b, a \} | = 2
\]

\phantom{The number of elements in the set $\{a,b,a\}$ is $2$.}

\[
    | aba | = 3
\]

\phantom{The length of the string $aba$ is $3$.}

\[
    (a, 3, 2, b, b)
\]

\phantom{The $5$-tuple whose first components is $a$, second component 
is $3$, third component is $2$, fourth component is $b$, and fifth component is $b$.}



\begin{center}
    \begin{tabular}{|p{2in}cp{4in}|}
    \hline 
    Term & Typical symbol & Meaning \\
    \hline\hline
    Alphabet & $\Sigma$, $\Gamma$ & A non-empty finite set	 \\ \hline
    Symbol over $\Sigma$  & $\sigma$, $b$, $x$ & An element of the alphabet $\Sigma$\\ \hline
    String over $\Sigma$  &	$u$, $v$, $w$ & A finite list of symbols from $\Sigma$\\ \hline
    The set of all strings over $\Sigma$ & $\Sigma^*$ & The collection of all possible strings formed from symbols from $\Sigma$ \\ \hline
    (Some) language over $\Sigma$& $L$ & (Some) set of strings over $\Sigma$ \\ \hline
    Empty string &$\varepsilon$ & The string of length $0$\\ \hline
    Empty set &$\emptyset$ & The empty language\\ \hline
    The power set of a set $X$ &$\mathcal{P}(X)$ & The set of all subsets of $X$ \\ \hline
    Natural numbers &$\mathcal{N}$ & The set of positive integers \\ \hline
    Finite set & & The empty set or a set whose distinct elements can be counted by a natural number\\ \hline
    Infinite set & & A set that is not finite.\\ 
    \hline\hline
    {\it Pages 3, 4, 6, 13, 14, 53}& & \\
    \hline
    \end{tabular}
\end{center}

\newpage

\begin{center}
    \begin{tabular}{|p{2.7in}cp{3.8in}|}
    \hline
    Term & Notation & Meaning \\
    \hline \hline
    Reverse of a string $w$ & $w^\mathcal{R}$  & write $w$  in  the opposite order, if $w = w_1 \cdots  w_n$ then $w^\mathcal{R} = w_n \cdots  w_1$. Note: $\varepsilon^\mathcal{R} = \varepsilon$\\ \hline
    Concatenating strings $x$ and $y$ & $xy$ &  take $x = x_1 \cdots x_m$, $y=y_1 \cdots y_n$ and form $xy = x_1 \cdots x_m y_1 \cdots y_n$\\ \hline
    String $z$ is a substring of string $w$ & & there are strings $u,v$ such that $w = uzv$\\ \hline
    String $x$ is a prefix of string $y$ & & there is a string $z$ such that $y = xz$ \\ \hline
    String $x$ is a proper prefix of string $y$ & & $x$ is a prefix of $y$ and $x \neq y$\\ \hline
    Shortlex order, also known as string order over alphabet $\Sigma$ & & Order strings over  $\Sigma$ first by length and then according to the dictionary order, assuming symbols in $\Sigma$  have an ordering.\\ \hline
    \hline \hline
    {\it Pages 13, 14} & & \\
    \hline
    \end{tabular}
\end{center}

    

{\it Circle the correct choice}:

A {\bf string} over an alphabet $\Sigma$ is \underline{~~an element of $\Sigma^*$ ~~ OR ~~ a subset of $\Sigma^*$}.
    
A {\bf language} over an alphabet $\Sigma$ is \underline{~~an element of $\Sigma^*$ ~~ OR ~~ a subset of $\Sigma^*$}.


{\it Extra examples for practice:}

With $\Sigma_1 = \{0,1\}$ and $\Sigma_2 = \{a,b,c,d,e,f,g,h,i,j,k,l,m,n,o,p,q,r,s,t,u,v,w,x,y,z\}$  and $\Gamma = \{0,1,x,y,z\}$

An example of a string of length 3 over $\Sigma_1$ is \underline{\phantom{ $000$} \hspace{0.2in}}

An example of  a string of length 1 over $\Sigma_2$ is  \underline{\phantom{ $k$} \hspace{0.2in}}

The number of distinct strings of length 2 over $\Gamma$ is  \underline{\phantom{ $25$} \hspace{0.2in}}

An example of a language over $\Sigma_1$ of size $1$ is  \underline{\phantom{ $ \{ \varepsilon \} $} \hspace{0.2in}}

An example of an infinite language over $\Sigma_1$ is  \underline{\phantom{ $\Sigma^*$} \hspace{0.2in}}
    
An example of  a finite language over $\Gamma$ is  \underline{\phantom{ $\{ 0, x \}$} \hspace{0.2in}}
    
{\bf True} or {\bf False}: $\varepsilon \in \Sigma_1$

{\bf True} or {\bf False}: $\varepsilon$ is  a string over $\Sigma_1$

{\bf True} or {\bf False}: $\varepsilon$ is a language over $\Sigma_1$

{\bf True} or {\bf False}: $\varepsilon$ is a prefix of some string over  $\Sigma_1$

{\bf True} or {\bf False}: There is a string over $\Sigma_1$ that is a proper prefix of $\varepsilon$
    

The first five strings over $\Sigma_1$ in string order, using the ordering $0 <  1$: \vfill
    
The first five strings over $\Sigma_2$ in string order, using the usual alphabetical ordering for single letters: \vfill
 \vfill
\section*{Week1 wednesday}


Our motivation in studying sets of strings is that they can be used to encode problems.
To calibrate how difficult a problem is to solve, we describe how complicated the set of strings that encodes it is. 
How do we define sets of strings?


\vfill

How would you describe the language that has no elements at all?

\vfill

How would you describe the language that has all strings over $\{0,1\}$ as its elements?

\vfill

\newpage


{\bf Definition 1.52}: A {\bf regular expression} over alphabet $\Sigma$
is a syntactic expression that can describe a language over $\Sigma$. The collection of all regular
expressions is defined recursively:
\begin{itemize}
\item[] {\it Basis steps of recursive definition}
\begin{quote}    
    $a$ is a regular expression, for $a \in \Sigma$

    $\varepsilon$ is a regular expression

    $\emptyset$ is a regular expression
\end{quote}

\item[] {\it Recursive steps of recursive definition}
\begin{quote}
    $(R_1 \cup R_2)$ is a regular expression when $R_1$, $R_2$ are regular expressions 

    $(R_1 \circ R_2)$ is a regular expression when $R_1$, $R_2$ are regular expressions

    $(R_1^*)$ is a regular expression when $R_1$ is a regular expression 
\end{quote}
\end{itemize}
 

The {\it semantics} (or meaning) of the syntactic regular expression is the {\bf language
described by the regular expression}. The function that assigns a language to a regular expression
over $\Sigma$ is defined recursively, using familiar set operations:


\begin{itemize}
    \item[] {\it Basis steps of recursive definition}
    \begin{quote}    
        The language described by $a$, for $a \in \Sigma$, is $\{a\}$ and we write 
        $L(a) = \{a\}$
    
        The language described by $\varepsilon$ is $\{\varepsilon\}$ and we write 
        $L(\varepsilon) = \{ \varepsilon\}$
    
        The language described by $\emptyset$ is $\{\}$ and we write
        $L(\emptyset) = \emptyset$.
    \end{quote}
    
    \item[] {\it Recursive steps of recursive definition}
    \begin{quote}
        When $R_1$, $R_2$ are regular expressions, the language described by the regular
        expression $(R_1 \cup R_2)$ is the union of the languages described by $R_1$ and $R_2$, 
        and we write 
        $$L(~(R_1 \cup R_2)~) = L(R_1) \cup L(R_2) = \{ w \mid w \in L(R_1) \lor w \in L(R_2)\}$$
    
        When $R_1$, $R_2$ are regular expressions, the language described by the regular
        expression $(R_1 \circ R_2)$ is the concatenation of the languages described by $R_1$ and $R_2$, 
        and we write 
        $$L(~(R_1 \circ R_2)~) = L(R_1) \circ L(R_2) = \{ uv \mid u \in L(R_1) \land v \in L(R_2)\}$$
    
        When $R_1$ is a regular expression, the language described by the regular 
        expression $(R_1^*)$ is the {\bf Kleene star} of the language described by $R_1$ and we write
        $$L(~(R_1^*)~) = (~L(R_1)~)^* = \{ w_1 \cdots w_k \mid k \geq 0 \textrm{ and each } w_i \in L(R_1)\}$$
    \end{quote}
\end{itemize}
  
For the following examples assume the alphabet is $\Sigma_1 =  \{0,1\}$:
    
The language described by the regular expression $0$ is $L(0) = \{ 0 \}$

The language described by the regular expression $1$ is $L(1)  = \{ 1 \}$

The language described by the regular expression $\varepsilon$ is $L(\varepsilon) = \{ \varepsilon  \}$

The language described by the regular expression $\emptyset$ is $L(\emptyset) = \emptyset$

The language described by the regular expression $((0 \cup 1) \cup 1)$ is $L(~((0 \cup 1) \cup 1)~) = $

\vfill

The language described by the regular expression $1^+$ is $L(~(1)^+~) = $

\vfill

The language described by the regular expression $\Sigma_1^* 1$ is $L(~\Sigma_1^* 1~) = $

\vfill

The language described by the regular expression $(\Sigma_1 \Sigma_1 \Sigma_1 \Sigma_1 \Sigma_1)^*$ 
is $L(~(\Sigma_1 \Sigma_1 \Sigma_1 \Sigma_1 \Sigma_1)^*~) = $

\vfill

A regular expression that describes the language $\{00, 01, 10, 11\}$ is 

\vfill

A regular expression that describes the language $\{ 0^n1 \mid n~\text{is even} \}$ is

\vfill

{\it Shorthand and conventions}
    
\begin{center}
    \begin{tabular}{|ll|}
    \hline
    \multicolumn{2}{|l|}{Assuming $\Sigma$ is the alphabet, we use the following conventions}\\
    \hline
    $\Sigma$   & regular  expression describing language consisting of  all strings  of length  $1$ over $\Sigma$\\
    $*$ then $\circ$ then $\cup$   & precedence order, unless parentheses are used to change it\\
    $R_1R_2$ & shorthand  for  $R_1  \circ R_2$ (concatenation symbol is implicit) \\
    $R^+$ & shorthand for $R^* \circ R$ \\
    $R^k$ & shorthand for $R$ concatenated with itself $k$ times, where $k$ is a natural number\\
    \hline
    {\it Pages 63 - 65 }& \\
    \hline
    \end{tabular}
\end{center}

\newpage
{\bf Caution: many programming languages that support regular expressions build in functionality
that is more powerful than the ``pure'' definition of regular expressions given here. }

Regular expressions are everywhere (once you start looking for them).

Software tools and languages often have built-in support for regular expressions to describe
{\bf patterns} that we want to match (e.g. Excel/ Sheets, grep, Perl, python, Java, Ruby).

Under the hood, the first phase of {\bf compilers} is to transform the strings we write 
in code to tokens (keywords, operators, identifiers, literals). Compilers use regular expressions
to describe the sets of strings that can be used for each token type.

Next time: we'll start to see how to build machines that decide whether strings match the pattern
described by a regular expression.

\vfill


{\it Extra examples for practice:}

Which regular expression(s) below describe a language that includes the string $a$ as an element?

$a^* b^*$ 

\vfill

$a(ba)^* b$

\vfill

$a^* \cup b^*$

\vfill

$(aaa)^*$

\vfill

$(\varepsilon \cup a) b$

\vfill \vfill
\section*{Week1 friday}


{\bf Review}: Determine whether each statement below about regular expressions
over the alphabet $\{a,b,c\}$ is true or false:
   
True or False: \qquad 
   $a  \in L(~(a \cup b )~\cup c)$

True or False: \qquad 
   $ab  \in L(~ (a \cup b)^*  ~)$
   
True or False: \qquad    
   $ba \in L( ~ a^* b^* ~)$
   
True or False: \qquad 
   $\varepsilon  \in L(a \cup b \cup c)$
   
True or False: \qquad 
   $\varepsilon  \in L(~ (a \cup b)^*  ~)$

True or False: \qquad 
   $\varepsilon \in L( ~ a^* b^* ~)$


{\bf From the pre-class reading, pages 34-36}:
A deterministic finite automaton (DFA) is specified by  $M = (Q, \Sigma, \delta, q_0, F)$.
This $5$-tuple is called the {\bf formal definition} of the DFA. The DFA can also 
be represented by its state diagram: with nodes for the state, labelled edges specifying the 
transition function, and decorations on nodes denoting the start and accept states.

\begin{quote}
Finite set of states $Q$ can be labelled by any collection of distinct names. Often
we use default state labels $q0, q1, \ldots$ 
\end{quote}

\begin{quote}  
The alphabet $\Sigma$ determines the possible inputs to the automaton. 
Each input to the automaton is a string over  $\Sigma$, and the automaton ``processes'' the input
one symbol (or character) at a time.
\end{quote}

\begin{quote}
The transition function $\delta$ gives the next state of the DFA based on the current state of 
the machine and on the next input symbol.
\end{quote}

\begin{quote}
The start state $q_0$ is an element of $Q$.  Each computation of the machine starts at the  start  state.
\end{quote}

\begin{quote}
The accept (final) states $F$ form a subset of the states of the DFA, $F \subseteq  Q$. 
These states are used to flag if the machine accepts or rejects an input string.
\end{quote}


\begin{quote}
The computation of a machine on an input string is a sequence of states
in the machine,  starting with the start state, determined by transitions 
of the machine as it reads successive input symbols.
\end{quote}

\begin{quote}
The DFA $M$ accepts the given input string exactly when the computation of $M$ on the input string
ends in an accept state. $M$ rejects the given input string exactly when the computation of 
$M$ on the input string ends in a nonaccept state, that is, a state that is not in $F$.
\end{quote}

\begin{quote} 
The language of $M$, $L(M)$, is defined as the set of  all strings that are each accepted 
by the machine $M$. Each string that is rejected by $M$ is not in $L(M)$.
The language of $M$ is also called the language recognized by $M$.
\end{quote}   
   
What is {\bf finite} about all deterministic finite automata? (Select all that apply)
\begin{itemize}
   \item[$\square$] The size of the machine (number of states, number of arrows)
   \item[$\square$] The number of strings that are accepted by the machine
   \item[$\square$] The length of each computation of the machine
\end{itemize}
  
\begin{figure}[h]
   \centering
   \includegraphics[width=3in]{../../resources/machines/Lect2DFA1.png} 
\end{figure}
   
The formal definition of this DFA is
   
\vspace{100pt}
   

Classify each string $a, aa, ab, ba, bb, \varepsilon$ as accepted by the DFA or rejected by the DFA.  

{\it Why are these the only two options?}

\vspace{200pt}


The language recognized by this DFA is
  
\vspace{100pt}
   

\begin{figure}[h]
  \centering
  \includegraphics[width=3in]{../../resources/machines/Lect2DFA2.png} 
\end{figure}
   

The language recognized by this DFA is
  
\vspace{100pt}

\begin{figure}[h]
    \centering
    \includegraphics[width=3in]{../../resources/machines/Lect2DFA3.png} 
\end{figure}

The language recognized by this DFA is
  
\vspace{100pt}
 \vfill
\end{document}