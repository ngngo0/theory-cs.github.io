\documentclass[12pt, oneside]{article}

\usepackage[letterpaper, scale=0.89, centering]{geometry}
\usepackage{fancyhdr}
\setlength{\parindent}{0em}
\setlength{\parskip}{1em}

\pagestyle{fancy}
\fancyhf{}
\renewcommand{\headrulewidth}{0pt}
\rfoot{\href{https://creativecommons.org/licenses/by-nc-sa/2.0/}{CC BY-NC-SA 2.0} Version \today~(\thepage)}

\usepackage{amssymb,amsmath,pifont,amsfonts,comment,enumerate,enumitem}
\usepackage{currfile,xstring,hyperref,tabularx,graphicx,wasysym}
\usepackage[labelformat=empty]{caption}
\usepackage{xcolor}
\usepackage{multicol,multirow,array,listings,tabularx,lastpage,textcomp,booktabs}

\lstnewenvironment{algorithm}[1][] {   
    \lstset{ mathescape=true,
        frame=tB,
        numbers=left, 
        numberstyle=\tiny,
        basicstyle=\rmfamily\scriptsize, 
        keywordstyle=\color{black}\bfseries,
        keywords={,procedure, div, for, to, input, output, return, datatype, function, in, if, else, foreach, while, begin, end, }
        numbers=left,
        xleftmargin=.04\textwidth,
        #1
    }
}
{}

\newcommand\abs[1]{\lvert~#1~\rvert}
\newcommand{\st}{\mid}

\newcommand{\cmark}{\ding{51}}
\newcommand{\xmark}{\ding{55}}
 
\begin{document}
\begin{flushright}
    \StrBefore{\currfilename}{.}
\end{flushright} \section*{Week3 friday}



{\bf Theorem}: For an alphabet $\Sigma$, For each language $L$ over $\Sigma$, 
\begin{center}
$L$ is recognized by some DFA \\
iff\\
$L$ is recognized by some NFA\\
iff\\
$L$ is described by some regular expression
\end{center}
If (any, hence all) these conditions apply, $L$ is called {\bf regular}.



{\bf Prove or Disprove}: There is some alphabet $\Sigma$ for which there is 
some language recognized by an NFA but not by any DFA.

\vspace{30pt}

{\bf Prove or Disprove}: There is some alphabet $\Sigma$ for which there is 
some finite language not described by any regular expression over $\Sigma$.

\vspace{30pt}


{\bf Prove or Disprove}: If a language is recognized by an NFA 
then the complement of this language is not recognized by any DFA.

\vspace{30pt}


\newpage
\begin{center}
\begin{tabular}{c|c}
Set & Cardinality \\
\hline
& \\
$\{0,1\}$ & \\
& \\
$\{0,1\}^*$ & \\
& \\
$\mathcal{P}( \{0,1\})$ & \\
& \\
The set of all languages over $\{0,1\}$ & \\
& \\
The set of all regular expressions over $\{0,1\}$ & \\
& \\
The set of all regular languages over $\{0,1\}$ & \\
& \\
\end{tabular}
\end{center}



\vfill

\newpage

{\bf Pumping Lemma} (Sipser Theorem 1.70): If $A$ is a regular language, then there
is a number $p$ (a {\it pumping length}) where, if $s$ is any string in $A$ of length at least $p$, 
then $s$ may be divided into three pieces, $s = xyz$ such that
\vspace{-10pt}
\begin{itemize}
\item $|y| > 0$
\item for each $i \geq 0$, $xy^i z \in A$
\item $|xy| \leq p$.
\end{itemize}


{\bf True or False}: A pumping length for $A = \{ 0,1 \}^*$ is $p = 5$.

\vspace{100pt}

{\bf True or False}: A pumping length for $A = \{1, 01, 001, 0001, 00001 \}$ is $p = 4$.

\vspace{100pt}

{\bf True or False}: A pumping length for $A = \{0^j 1 \mid  j \geq 0 \}$ is $p = 3$.


\vspace{100pt}

{\bf True or False}: For any language $A$, if $p$  is a  pumping length for $A$ and $p' > p$,  then 
$p'$ is also a pumping length for $A$.
 \vfill
\end{document}