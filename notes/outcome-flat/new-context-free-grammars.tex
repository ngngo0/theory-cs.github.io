\documentclass[12pt, oneside]{article}

\usepackage[letterpaper, scale=0.89, centering]{geometry}
\usepackage{fancyhdr}
\setlength{\parindent}{0em}
\setlength{\parskip}{1em}

\pagestyle{fancy}
\fancyhf{}
\renewcommand{\headrulewidth}{0pt}
\rfoot{\href{https://creativecommons.org/licenses/by-nc-sa/2.0/}{CC BY-NC-SA 2.0} Version \today~(\thepage)}

\usepackage{amssymb,amsmath,pifont,amsfonts,comment,enumerate,enumitem}
\usepackage{currfile,xstring,hyperref,tabularx,graphicx,wasysym}
\usepackage[labelformat=empty]{caption}
\usepackage[dvipsnames,table]{xcolor}
\usepackage{multicol,multirow,array,listings,tabularx,lastpage,textcomp,booktabs}

\lstnewenvironment{algorithm}[1][] {   
    \lstset{ mathescape=true,
        frame=tB,
        numbers=left, 
        numberstyle=\tiny,
        basicstyle=\rmfamily\scriptsize, 
        keywordstyle=\color{black}\bfseries,
        keywords={,procedure, div, for, to, input, output, return, datatype, function, in, if, else, foreach, while, begin, end, }
        numbers=left,
        xleftmargin=.04\textwidth,
        #1
    }
}
{}
\lstnewenvironment{java}[1][]
{   
    \lstset{
        language=java,
        mathescape=true,
        frame=tB,
        numbers=left, 
        numberstyle=\tiny,
        basicstyle=\ttfamily\scriptsize, 
        keywordstyle=\color{black}\bfseries,
        keywords={, int, double, for, return, if, else, while, }
        numbers=left,
        xleftmargin=.04\textwidth,
        #1
    }
}
{}

\newcommand\abs[1]{\lvert~#1~\rvert}
\newcommand{\st}{\mid}

\newcommand{\A}[0]{\texttt{A}}
\newcommand{\C}[0]{\texttt{C}}
\newcommand{\G}[0]{\texttt{G}}
\newcommand{\U}[0]{\texttt{U}}

\newcommand{\cmark}{\ding{51}}
\newcommand{\xmark}{\ding{55}}
 
\begin{document}
\begin{flushright}
    \StrBefore{\currfilename}{.}
\end{flushright} \section*{Week5 monday}


To fully specify a PDA we could give its $6$-tuple formal definition or we could give its input 
alphabet, stack alphabet, and state diagram.
An informal description of a PDA is a step-by-step description of how its computations 
would process input strings; the reader should be able to reconstruct the state diagram or formal 
definition precisely from such a descripton. The informal description of a PDA can refer to some 
common modules or subroutines that are computable by PDAs:
\begin{itemize}
  \item PDAs can ``test for emptyness of stack'' without providing details. 
  {\it How?} We can always push a special end-of-stack symbol, $\$$, at the start, before processing
  any input, and then use this symbol as a flag.
  \item PDAs can ``test for end of input'' without providing details.
  {\it How?} We can transform a PDA to one where accepting states are only those reachable 
  when there are no more input symbols.
\end{itemize}

{\it Big picture}: PDAs were motivated by wanting to add some memory of unbounded size to NFA. How 
do we accomplish a similar enhancement of regular expressions to get a syntactic model that is 
more expressive?

DFA, NFA, PDA: Machines process one input string at a time; the computation of a machine on its input string 
reads the input from left to right.

Regular expressions: Syntactic descriptions of all strings that match a particular pattern; the language 
described by a regular expression is built up recursively according to the expression's syntax

{\bf Context-free grammars}: Rules to produce one string at a time, adding characters from the middle, beginning, 
or end of the final string as the derivation proceeds.


\begin{center}
  \hspace{-0.25in}\begin{tabular}{|p{2in}cp{4in}|}
  \hline 
  Term & Typical symbol & Definition \\
  \hline\hline
  {\bf Context-free grammar} (CFG) & $G$ & $G = (V, \Sigma, R, S)$ \\
  {\bf Variables}& $V$ & Finite  set of symbols that represent phases in production pattern\\
  {\bf Terminals} & $\Sigma$ & Alphabet of symbols of strings generated  by CFG \\
  & & $V \cap \Sigma = \emptyset$ \\
  {\bf Rules}& $R$ & Each rule is  $A \to u$ with $A \in V$ and $u  \in (V  \cup \Sigma)^*$\\
  Start variable&  $S$  & Usually  on LHS of first / topmost rule \\
  {\bf Derivation} & & Sequence  of substitutions in a  CFG \\
  & $S \implies \cdots \implies w$ & Start with start variable, apply one rule to one occurrence of a variable at a time\\
  {\bf Language} generated by the CFG $G$ & $L(G)$ &$\{  w \in \Sigma^* \mid \text{there is  derivation in $G$ that ends
  in $w$} \} = \{  w \in \Sigma^* \mid S \implies^* w \}$\\
  {\bf Context-free language} & & A language that is the language generated by some CFG\\
  \hline
  Sipser pages 102-103 & &\\
  \hline
  \end{tabular}
  \end{center}
  
{\bf Examples of context-free grammars, derivations in those grammars, and the languages generated by those grammars}
  
$G_1 =  (\{S\}, \{0\}, R, S)$ with rules
  \begin{align*}
    &S \to 0S\\
    &S \to 0\\
  \end{align*}
  In  $L(G_1)$ \ldots 
  
  \vspace{110pt}
  
  Not in $L(G_1)$ \ldots 

  \vspace{110pt}


  $G_2 =  (\{S\}, \{0,1\}, R, S)$
  \[
  S \to 0S \mid 1S \mid \varepsilon
  \]
  In  $L(G_2)$ \ldots 
  
  \vspace{110pt}
  
  Not in $L(G_2)$ \ldots 

  \vspace{110pt}

  $(\{S, T\}, \{0, 1\}, R, S)$ with  rules
  \begin{align*}
  &S \to T1T1T1T \\
  &T \to  0T \mid 1T \mid \varepsilon
  \end{align*}

  In  $L(G_3)$ \ldots 
  
  \vspace{110pt}
  
  Not in $L(G_3)$ \ldots 

  \vspace{110pt}


  $G_4 =  (\{A, B\}, \{0, 1\}, R, A)$ with rules
  \[
    A \to 0A0 \mid  0A1 \mid 1A0  \mid 1A1 \mid  1
  \]
  In  $L(G_4)$ \ldots 
  
  \vspace{110pt}
  
  Not in $L(G_4)$ \ldots 

  \vspace{110pt}

  
{\it Extra practice}: Is there a CFG $G$ with $L(G) = \emptyset$?

  \newpage
  
  Design a CFG to generate the  language $\{abba\}$
  
  \begin{align*}
  & ( \{ S, T, V, W\}, \{a,b\}, \{ S \to aT, T \to bV, V \to bW, W \to a\}, S)\\
  & \\ 
  & \\ 
  & \\ 
  & ( \{ Q \}, \{a,b\}, \{Q \to abba\}, Q) \\
  & \\ 
  & \\ 
  & \\
  & ( \{ X,Y \}, \{a,b\}, \{X \to aYa, Y \to bb\}, X) 
  & \\ 
  & \\ 
  \end{align*} 


  Design a CFG to generate the  language $\{a^n b^n \mid  n  \geq  0\}$
  
  \vspace{150pt}
  
  {\it Sample derivation:} 
  
  \vspace{30pt}
  
  
  Design a CFG to generate the  language $\{a^i b^j \mid j \geq i  \geq 0\}$
  
  \vspace{150pt}
  
  {\it Sample derivation:} 
  
  \vspace{30pt}
 \vfill
\section*{Week5 wednesday}


{\bf Theorem  2.20}: A language is  generated by some context-free  grammar
if  and only if it is recognized by some push-down automaton.

Definition: a language is called {\bf context-free} if it is the language generated by a context-free grammar.
The class of all context-free language over a given alphabet $\Sigma$ is called {\bf CFL}.

Consequences:
\begin{itemize}
    \item Quick proof that every regular language is context free 
    \item To prove closure of the class of context-free languages under a given operation, we can choose 
    either of two modes 
    of proof (via CFGs or PDAs) depending on which is easier
\end{itemize}


Over $\Sigma = \{a,b\}$, let $L = \{ a^n b^m \mid n  \neq m \}$. {\bf Goal}: Prove $L$ is context-free.


\vfill

\newpage
Suppose $L_1$ and $L_2$ are context-free languages over $\Sigma$.  {\bf Goal}:  $L_1 \cup L_2$  is  also context-free.

{\it Approach 1: with  PDAs}

Let $M_1 = ( Q_1, \Sigma, \Gamma_1, \delta_1, q_1, F_1)$ and
$M_2 = ( Q_2, \Sigma, \Gamma_2, \delta_2, q_2, F_2)$ be PDAs with 
$L(M_1) =  L_1$  and  $L(M_2) = L_2$.

Define $M = $

\vfill

{\it Approach  2: with CFGs}

Let $G_1 = (V_1, \Sigma, R_1, S_1)$  and   $G_2 = (V_2, \Sigma, R_2, S_2)$  be CFGs  with
$L(G_1) =  L_1$  and  $L(G_2) = L_2$.

Define $G = $

\vfill

\newpage
Suppose $L_1$ and $L_2$ are context-free languages over $\Sigma$.  {\bf Goal}:  $L_1 \circ L_2$  is  also context-free.


{\it Approach 1: with  PDAs}

Let $M_1 = ( Q_1, \Sigma, \Gamma_1, \delta_1, q_1, F_1)$ and
$M_2 = ( Q_2, \Sigma, \Gamma_2, \delta_2, q_2, F_2)$ be PDAs with 
$L(M_1) =  L_1$  and  $L(M_2) = L_2$.

Define $M = $

\vfill

{\it Approach  2: with CFGs}

Let $G_1 = (V_1, \Sigma, R_1, S_1)$  and   $G_2 = (V_2, \Sigma, R_2, S_2)$  be CFGs  with
$L(G_1) =  L_1$  and  $L(G_2) = L_2$.

Define $G = $

\vfill
\newpage
{\it Summary}

Over a fixed alphabet $\Sigma$, a language $L$ is {\bf regular}

\vspace{-20pt}
\begin{center}
    iff it is described by some regular expression \\
    iff it is recognized by some DFA\\
    iff it is recognized by some NFA
\end{center}

Over a fixed alphabet $\Sigma$, a language $L$ is {\bf context-free}

\vspace{-20pt}
\begin{center}
    iff it is generated by some CFG\\
    iff it is recognized by some PDA
\end{center}

{\bf Fact}: Every regular language is a context-free language.

{\bf Fact}: There are context-free languages that are not nonregular.

{\bf Fact}: There are countably many regular languages.

{\bf Fact}: There are countably inifnitely many context-free languages.

{\it Consequence}: Most languages are {\bf not} context-free!

{\bf Examples  of non-context-free languages}

\begin{align*}
    &\{ a^n b^n c^n \mid 0 \leq n , n \in \mathbb{Z}\}\\
    &\{ a^i b^j c^k \mid 0 \leq i \leq j \leq k , i \in \mathbb{Z}, j \in \mathbb{Z}, k \in \mathbb{Z}\}\\
    &\{ ww \mid w \in \{0,1\}^* \}
\end{align*}
(Sipser Ex 2.36, Ex 2.37, 2.38)

There is a Pumping Lemma for CFL that can be used to prove a specific language is non-context-free: 
If $A$ is a context-free language, there there
is a number $p$ where, if $s$ is any string in $A$ of length at least $p$, then $s$ may be divided 
into five pieces $s = uvxyz$ where (1) for each $i \geq 0$, $uv^ixy^iz \in A$, (2) $|uv|>0$, (3) $|vxy| \leq p$.
{\it We will not go into the details of the proof or application of Pumping Lemma for CFLs this quarter.} \vfill
\end{document}