\documentclass[12pt, oneside]{article}

\usepackage[letterpaper, scale=0.89, centering]{geometry}
\usepackage{fancyhdr}
\setlength{\parindent}{0em}
\setlength{\parskip}{1em}

\usepackage{tikz}
\usetikzlibrary{automata,positioning,arrows}

\pagestyle{fancy}
\fancyhf{}
\renewcommand{\headrulewidth}{0pt}
\rfoot{\href{https://creativecommons.org/licenses/by-nc-sa/2.0/}{CC BY-NC-SA 2.0} Version \today~(\thepage)}

\usepackage{amssymb,amsmath,pifont,amsfonts,comment,enumerate,enumitem}
\usepackage{currfile,xstring,hyperref,tabularx,graphicx,wasysym}
\usepackage[labelformat=empty]{caption}
\usepackage{xcolor}
\usepackage{multicol,multirow,array,listings,tabularx,lastpage,textcomp,booktabs}

\lstnewenvironment{algorithm}[1][] {   
    \lstset{ mathescape=true,
        frame=tB,
        numbers=left, 
        numberstyle=\tiny,
        basicstyle=\rmfamily\scriptsize, 
        keywordstyle=\color{black}\bfseries,
        keywords={,procedure, div, for, to, input, output, return, datatype, function, in, if, else, foreach, while, begin, end, }
        numbers=left,
        xleftmargin=.04\textwidth,
        #1
    }
}
{}

\newcommand\abs[1]{\lvert~#1~\rvert}
\newcommand{\st}{\mid}

\newcommand{\cmark}{\ding{51}}
\newcommand{\xmark}{\ding{55}}
 
\begin{document}
\begin{flushright}
    \StrBefore{\currfilename}{.}
\end{flushright} \section*{Week8 monday}



{\bf  Theorem}: $A_{TM}$  is  not  Turing-decidable.

{\bf  Proof}: Suppose {\bf towards a  contradiction}  that there  is a Turing machine  that decides $A_{TM}$.  
We call this presumed machine  $M_{ATM}$.

By  assumption, for every  Turing machine  $M$ and every  string $w$

\begin{itemize}
\item If $w \in L(M)$, then  the computation of $M_{ATM}$  on  $\langle M,w \rangle ~~ \underline{\phantom{\hspace{2.5in}}}$
\item If $w \notin L(M)$, then  the computation of $M_{ATM}$  on  $\langle M,w \rangle ~~ \underline{\phantom{\hspace{2.5in}}}$
\end{itemize}


Define  a {\bf new} Turing machine using  the high-level description:
\begin{quote}
$D =  $`` On  input $\langle M \rangle$, where  $M$  is  a Turing machine:
\begin{itemize}
\item[1.] Run  $M_{ATM}$ on  $\langle M, \langle M \rangle  \rangle$.
\item[2.] If $M_{ATM}$ accepts, reject; if  $M_{ATM}$ rejects, accept."
\end{itemize}
\end{quote}


Is $D$ a  Turing machine?

\vspace{50pt}

Is  $D$ a  decider? 

\vspace{50pt}

What is the result of the computation  of $D$  on  $\langle D \rangle$?

\vfill

\newpage

{\bf  Theorem} (Sipser Theorem 4.22): A  language is Turing-decidable if and only if both  it and its complement
are Turing-recognizable.

{\bf Proof, first direction:}  Suppose  language  $L$ is  Turing-decidable.   WTS  that both it and its complement 
are Turing-recognizable.

\vfill

{\bf Proof, second direction:}  Suppose  language  $L$ is  Turing-recognizable, and  so is  its complement.   WTS  that $L$
is Turing-decidable.
\vfill


Give an example of a {\bf decidable} set: 

\vspace{20pt}

Give an example of a {\bf recognizable undecidable} set: 

\vspace{20pt}

Give an example of an {\bf unrecognizable} set: 

\vspace{20pt}


\newpage

{\bf True} or {\bf False}: The class of Turing-decidable languages is closed under complementation?

\vfill
\vfill
\vfill
Definition: A language $L$ over an  alphabet $\Sigma$ is called {\bf co-recognizable} if its complement,  defined
as $\Sigma^* \setminus L  = \{ x  \in  \Sigma^* \mid x \notin  L \}$, is Turing-recognizable.


Notation: The complement  of a set $X$ is denoted with  a superscript $c$, $X^c$, or an overline,  $\overline{X}$.
 \vfill
\end{document}