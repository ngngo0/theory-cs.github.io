\documentclass[12pt, oneside]{article}

\usepackage[letterpaper, scale=0.89, centering]{geometry}
\usepackage{fancyhdr}
\setlength{\parindent}{0em}
\setlength{\parskip}{1em}

\usepackage{tikz}
\usetikzlibrary{automata,positioning,arrows}

\pagestyle{fancy}
\fancyhf{}
\renewcommand{\headrulewidth}{0pt}
\rfoot{\href{https://creativecommons.org/licenses/by-nc-sa/2.0/}{CC BY-NC-SA 2.0} Version \today~(\thepage)}

\usepackage{amssymb,amsmath,pifont,amsfonts,comment,enumerate,enumitem}
\usepackage{currfile,xstring,hyperref,tabularx,graphicx,wasysym}
\usepackage[labelformat=empty]{caption}
\usepackage{xcolor}
\usepackage{multicol,multirow,array,listings,tabularx,lastpage,textcomp,booktabs}

\lstnewenvironment{algorithm}[1][] {   
    \lstset{ mathescape=true,
        frame=tB,
        numbers=left, 
        numberstyle=\tiny,
        basicstyle=\rmfamily\scriptsize, 
        keywordstyle=\color{black}\bfseries,
        keywords={,procedure, div, for, to, input, output, return, datatype, function, in, if, else, foreach, while, begin, end, }
        numbers=left,
        xleftmargin=.04\textwidth,
        #1
    }
}
{}

\newcommand\abs[1]{\lvert~#1~\rvert}
\newcommand{\st}{\mid}

\newcommand{\cmark}{\ding{51}}
\newcommand{\xmark}{\ding{55}}
 
\begin{document}
\begin{flushright}
    \StrBefore{\currfilename}{.}
\end{flushright} 
\section*{Monday}



Recall Definition  (Sipser 7.1): For  $M$ a deterministic decider, its {\bf running time} is the function  $f: \mathbb{N} \to \mathbb{N}$
given  by
\[
f(n) =  \text{max number of  steps $M$ takes before halting, over all inputs  of length $n$}
\]

Recall Definition (Sipser 7.7): For each function $t(n)$, the {\bf time complexity class}  $TIME(t(n))$, is defined  by
\[
TIME( t(n)) = \{ L \mid \text{$L$ is decidable by  a Turing machine with running time in  $O(t(n))$} \}
\]
Recall Definition (Sipser 7.12) : $P$ is the class of languages that  are decidable in polynomial time on 
a deterministic 1-tape  Turing  machine
\[
P  =  \bigcup_k TIME(n^k)
\]

Definition (Sipser  7.9): For $N$ a nodeterministic decider.  
The {\bf running time} of $N$ is the function $f: \mathbb{N} \to \mathbb{N}$ given  by
\[
f(n) =  \text{max number of  steps $N$ takes on  any branch before halting, over all inputs  of length $n$}
\]

Definition (Sipser 7.21): For each function $t(n)$, the {\bf nondeterministic time complexity class}  
$NTIME(t(n))$, is defined  by
\[
NTIME( t(n)) = \{ L \mid \text{$L$ is decidable by a nondeterministic Turing machine with running time in $O(t(n))$} \}
\]
\[
NP = \bigcup_k NTIME(n^k)
\]


{\bf True} or {\bf False}: $TIME(n^2) \subseteq NTIME(n^2)$

\vfill

{\bf True} or {\bf False}: $NTIME(n^2) \subseteq DTIME(n^2)$

\vfill
\newpage

{\bf Examples in $P$ }

{\it Can't use nondeterminism; Can use multiple tapes; Often need to be “more clever” than naïve / brute force approach}
\[
    PATH = \{\langle G,s,t\rangle \mid \textrm{$G$ is digraph with $n$ nodes there is path from s to t}\}
\]
Use breadth first search to show in $P$
\[
    RELPRIME = \{ \langle x,y\rangle \mid \textrm{$x$ and $y$ are relatively prime integers}\}
\]
Use Euclidean Algorithm to show in $P$
\[
    L(G) = \{w \mid \textrm{$w$ is generated by $G$}\} 
\]
(where $G$ is a context-free grammar). Use dynamic programming to show in $P$.

\vfill
{\bf Examples in $NP$}

{\it ``Verifiable" i.e. NP, Can be decided by a nondeterministic TM in polynomial time,
best known deterministic solution may be brute-force, 
solution can be verified by a deterministic TM in polynomial time.}

\[
    HAMPATH = \{\langle G,s,t \rangle \mid \textrm{$G$ is digraph with $n$ nodes, there is path
from $s$ to $t$ that goes through every node exactly once}\}
\]
\[
    VERTEX-COVER = \{ \langle G,k\rangle \mid \textrm{$G$ is an undirected graph with $n$
nodes that has a $k$-node vertex cover}\}
\]
\[
    CLIQUE = \{ \langle G,k\rangle \mid \textrm{$G$ is an undirected graph with $n$ nodes that has a $k$-clique}\}
\]
\[
    SAT =\{ \langle X \rangle \mid \textrm{$X$ is a satisfiable Boolean formula with $n$ variables}\}
\]

\vfill
\newpage
{\bf Every problem in NP is decidable with an exponential-time algorithm}

Nondeterministic approach: guess a possible solution, verify that it works.

Brute-force (worst-case exponential time) approach: iterate over all possible solutions, for each 
one, check if it works.



\begin{center}
\begin{tabular}{c|c}
    {\bf Problems in $P$} & {\bf Problems in $NP$}\\
    \hline
    (Membership in any) regular language & Any problem in $P$ \\
    (Membership in any) context-free language &  \\
    $A_{DFA}$ & $SAT$\\
    $E_{DFA}$ & $CLIQUE$ \\
    $EQ_{DFA}$ & $VERTEX-COVER$ \\
    $PATH$ & $HAMPATH$ \\
    $RELPRIME$ &  $\ldots$ \\
$\ldots$ &\\
\end{tabular}
\end{center}

Million-dollar question: Is $P = NP$?


One approach to trying to answer it is to look for {\it hardest} problems in $NP$ and 
then (1) if we can show that there are efficient algorithms for them, then we can get 
efficient algorithms for all problems in $NP$ so $P = NP$, or (2) these problems might 
be good candidates for showing that there are problems in $NP$ for which there 
are no efficient algorithms.

\vfill
\newpage
 
\subsection*{Review: Week 10 Monday}

Please complete the review quiz questions on \href{http://gradescope.com}{Gradescope} about 
complexity ($P$ and $NP$)


\newpage
\subsection*{Wednesday}




Definition (Sipser 7.29) Language  $A$ is {\bf polynomial-time mapping reducible} to language $B$, written $A \leq_P B$,
means there is a polynomial-time computable function $f: \Sigma^* \to \Sigma^*$  such that for every $x \in \Sigma^*$
\[
x \in A \qquad \text{iff} \qquad f(x) \in B.
\]
The  function $f$ is  called the  polynomial time reduction of $A$ to $B$.

{\bf  Theorem}  (Sipser 7.31):  If $A \leq_P B$ and $B  \in P$ then $A \in P$.

Proof: 

\vfill

Definition (Sipser 7.34; based in Stephen Cook and Leonid Levin's work in the 1970s): 
A language $B$ is {\bf  NP-complete} means (1) $B$ is in NP {\bf and}  (2) every language
$A$ in $NP$ is polynomial time reducible to $B$.

{\bf  Theorem}  (Sipser 7.35):  If $B$ is NP-complete and $B \in P$ then $P = NP$.

Proof: 

\vfill

\newpage

{\bf 3SAT}: A literal is a Boolean variable (e.g.  $x$) or a negated Boolean variable (e.g.  $\bar{x}$).  
A Boolean formula is a {\bf  3cnf-formula} if it is a Boolean formula in conjunctive normal form (a conjunction  
of  disjunctive clauses of literals) and each clause  has  three literals.
\[
3SAT  = \{  \langle  \phi \rangle \mid \text{$\phi$ is  a  satisfiable 3cnf-formula} \}
\]


Example strings  in $3SAT$
\vfill



Example  strings not  in $3SAT$

\vfill




{\bf Cook-Levin Theorem}: $3SAT$ is $NP$-complete.


{\it Are there other $NP$-complete problems?} To prove that $X$ is $NP$-complete
\begin{itemize}
\item {\it From scratch}: prove $X$ is in $NP$ and that all $NP$ problems are polynomial-time
reducible to $X$.
\item {\it Using reduction}: prove $X$ is in $NP$ and that a known-to-be $NP$-complete problem 
is polynomial-time reducible to $X$.
\end{itemize}

\vfill
\vfill


\newpage

{\bf CLIQUE}: A {\bf $k$-clique} in an undirected graph is a maximally connected subgraph with $k$  nodes.
\[
CLIQUE  = \{  \langle G, k \rangle \mid \text{$G$ is an  undirected graph with  a $k$-clique} \}
\]


Example strings  in $CLIQUE$

\vfill

Example  strings not  in $CLIQUE$

\vfill

Theorem (Sipser 7.32):
\[
3SAT  \leq_P CLIQUE
\]

Given a Boolean formula in conjunctive normal form with $k$ clauses and three literals per clause, we will 
map it to a graph so that the graph has a clique if the original formula is satisfiable and the 
graph does not have a clique if the original formula is not satisfiable.

The graph has $3k$ vertices (one for each literal in each clause) and an edge between all vertices except
\begin{itemize}
    \item vertices for two literals in the same clause
    \item vertices for literals that are negations of one another
\end{itemize}

Example: $(x \vee \bar{y} \vee {\bar z}) \wedge (\bar{x}  \vee y  \vee  z) \wedge (x \vee y  \vee z)$

\vfill

\newpage
 
\subsection*{Review: Week 10 Wednesday}

Please complete the review quiz questions on \href{http://gradescope.com}{Gradescope} about 
complexity ($NP$-completeness)

\newpage
\subsection*{Friday}



\begin{center}
    \begin{tabular}{|p{4in}|p{3.5in}|}
        \hline
        & \\
        {\bf Model of Computation} & {\bf Class of Languages}\\
        &\\
        \hline
        & \\
        {\bf Deterministic finite automata}:
        formal definition, how to design for a given language, 
        how to describe language of a machine?
        {\bf Nondeterministic finite automata}:
        formal definition, how to design for a given language, 
        how to describe language of a machine?
        {\bf Regular expressions}: formal definition, how to design for a given language, 
        how to describe language of expression?
        {\it Also}: converting between different models. &
        {\bf Class of regular languages}: what are the closure 
        properties of this class? which languages are not in the class?
        using {\bf pumping lemma} to prove nonregularity.\\
        & \\
        \hline
        & \\
        {\bf Push-down automata}:
        formal definition, how to design for a given language, 
        how to describe language of a machine?
        {\bf Context-free grammars}:
        formal definition, how to design for a given language, 
        how to describe language of a grammar? &
        {\bf Class of context-free languages}: what are the closure 
        properties of this class? which languages are not in the class?\\
        & \\
        \hline
        & \\
        Turing machines that always halt in polynomial time
        & $P$ \\
        & \\
        Nondeterministic Turing machines that always halt in polynomial time 
        & $NP$ \\
        & \\
        \hline
        & \\
        {\bf Deciders} (Turing machines that always halt): 
        formal definition, how to design for a given language, 
        how to describe language of a machine? &
        {\bf Class of decidable languages}: what are the closure properties 
        of this class? which languages are not in the class? using diagonalization
        and mapping reduction to show undecidability \\
        & \\
        \hline
        & \\
        {\bf Turing machines}
        formal definition, how to design for a given language, 
        how to describe language of a machine? &
        {\bf Class of recognizable languages}: what are the closure properties 
        of this class? which languages are not in the class? using closure
        and mapping reduction to show unrecognizability \\
        & \\
        \hline
    \end{tabular}
\end{center}

\newpage

{\bf Given a language, prove it is regular}

{\it Strategy 1}: construct DFA recognizing the language and prove it works.

{\it Strategy 2}: construct NFA recognizing the language and prove it works.

{\it Strategy 3}: construct regular expression recognizing the language and prove it works.

{\it ``Prove it works'' means \ldots}

\vspace{100pt}

{\bf Example}: $L  = \{ w \in \{0,1\}^* \mid \textrm{$w$ has odd number of $1$s or starts with $0$}\}$

Using NFA

\vfill

Using regular expressions

\vfill


\newpage

{\bf Example}: Select all and only the options that result in a true statement: ``To show 
a language $A$ is not regular, we can\ldots'' 

\begin{enumerate}
    \item[a.] Show $A$ is finite
    \item[b.] Show there is a CFG generating $A$
    \item[c.] Show $A$ has no pumping length
    \item[d.] Show $A$ is undecidable
\end{enumerate}

\newpage

{\bf Example}: What is the language generated by the CFG with rules
\begin{align*}
    S &\to aSb \mid bY \mid Ya \\
    Y &\to bY \mid Ya \mid \varepsilon 
\end{align*}

\newpage

{\bf Example}: Prove that the language 
$T = \{ \langle M \rangle \mid \textrm{$M$ is a Turing machine and $L(M)$ is infinite}\}$ 
is undecidable.

\newpage

{\bf Example}: Prove that the class of decidable languages is closed under concatenation.
 
\newpage


\vfill

\begin{center}
\includegraphics[width=5in]{../../resources/images/wood-951875_960_720.jpeg}
\end{center}

\vfill

\subsection*{Review: Week 10 Friday}

Please complete the review quiz questions on \href{http://gradescope.com}{Gradescope} giving feedback on the quarter. 
Have a great summer!
\end{document}