\documentclass[12pt, oneside]{article}

\usepackage[letterpaper, scale=0.89, centering]{geometry}
\usepackage{fancyhdr}
\setlength{\parindent}{0em}
\setlength{\parskip}{1em}

\pagestyle{fancy}
\fancyhf{}
\renewcommand{\headrulewidth}{0pt}
\rfoot{\href{https://creativecommons.org/licenses/by-nc-sa/2.0/}{CC BY-NC-SA 2.0} Version \today~(\thepage)}

\usepackage{amssymb,amsmath,pifont,amsfonts,comment,enumerate,enumitem}
\usepackage{currfile,xstring,hyperref,tabularx,graphicx,wasysym}
\usepackage[labelformat=empty]{caption}
\usepackage{xcolor}
\usepackage{multicol,multirow,array,listings,tabularx,lastpage,textcomp,booktabs}

\lstnewenvironment{algorithm}[1][] {   
    \lstset{ mathescape=true,
        frame=tB,
        numbers=left, 
        numberstyle=\tiny,
        basicstyle=\rmfamily\scriptsize, 
        keywordstyle=\color{black}\bfseries,
        keywords={,procedure, div, for, to, input, output, return, datatype, function, in, if, else, foreach, while, begin, end, }
        numbers=left,
        xleftmargin=.04\textwidth,
        #1
    }
}
{}

\newcommand\abs[1]{\lvert~#1~\rvert}
\newcommand{\st}{\mid}

\newcommand{\cmark}{\ding{51}}
\newcommand{\xmark}{\ding{55}}
 
\begin{document}
\begin{flushright}
    \StrBefore{\currfilename}{.}
\end{flushright} 

In Computer Science, we operationalize ``hardest'' as ``requires most resources'', where
resources might be memory, time, parallelism, randomness, power, etc.
To be able to compare ``hardness'' of problems, we use a consistent description of problems

{\bf Input}: String

{\bf Output}: Yes/ No, where Yes means that the input string matches the pattern or property described by the problem.

So far: we saw that regular expressions are convenient ways of describring patterns in strings.
{\bf Finite automata} give a model of computation for processing strings and and classifying them into Yes (accepted)
or No (rejected). We will see that each set of strings is described by a regular expression if and only 
if there is a FA that recognizes it.  Another way of thinking about it: properties described by regular
expressions require exactly the computational power of these finite automata.


\subsection*{Wednesday: Finite automaton constructions}




{\bf Review}: Formal definition of finite automaton: $M = (Q, \Sigma, \delta, q_0, F)$ 

\begin{center}
\begin{multicols}{2}
\begin{itemize}
\setlength{\itemsep}{2pt}
\item Finite set of states $Q$
\item Alphabet $\Sigma$
\item Transition function $\delta$
\item Start state $q_0$
\item Accept (final) states $F$
\end{itemize}
\end{multicols}
\end{center}
In the state diagram of $M$, how many outgoing arrows are there from each state?

$M = ( \{ q, r, s\}, \{a,b\}, \delta, q, \{q\} )$ 
where $\delta$ is  (rows labelled by states
and columns labelled by symbols):
\begin{center}
\begin{tabular}{c|cc}
$\delta$ & $a$ & $b$ \\
\hline
$q$ & $r$ & $r$ \\
$r$ & $s$ & $s$ \\
$s$ & $q$ & $q$ \\
\end{tabular}
\end{center}

The state diagram for $M$ is 

\vfill



Give two examples of strings that are accepted by $M$ and two examples of strings that are rejected by $M$:

\vfill



$L(M) = $

A regular expression describing $L(M)$ is




\newpage
Let the alphabet be $\Sigma_1 = \{0,1\}$.

A state diagram for a finite automaton that recognizes $\{w \in \Sigma_1^* \mid w~\text{contains at most two $1$'s} \}$ is

\vfill

A state diagram for a finite automaton that recognizes $\{w \in \Sigma_1^* \mid w~\text{contains more than two $1$'s} \}$ is

\vfill

\textbf{Strategy}: Add ``labels" for states in the state diagram, e.g. 
``have not seen any of desired pattern yet'' or ``sink state''.
Then, we can use the analysis of the roles of the states in the state diagram to work towards a 
description of the language recognized by the finite automaton. 

\vfill
A useful bit of terminology: the {\bf iterated transition function} of a finite automaton
$M = (Q, \Sigma, \delta, q_0, F)$ is defined recursively by
\[
\delta^* (~(q,w)~) 
=\begin{cases}
q  \qquad &\text{if $q \in Q, w = \varepsilon$} \\
\delta( ~(q,a)~) \qquad &\text{if $q \in Q$, $w = a \in \Sigma$ } \\
\delta(~(\delta^*(q,u), a) ~) \qquad &\text{if $q \in Q$, $w = ua$ where $u \in  \Sigma^*$ and $a \in \Sigma$}
\end{cases}
\]

Using  this terminology, $M$ accepts a string $w$ over $\Sigma$ if and only if $\delta^*( ~(q_0,w)~) \in F$.

\newpage
Suppose $A$ is a language over an alphabet $\Sigma$. By definition, this means $A$ is a subset of $\Sigma^*$.
{\bf Claim:} if there is a DFA $M$ such that $L(M) = A$ then there is another DFA, let's call it $M'$, such that 
$L(M') = \overline{A}$, the complement of $A$, defined as $\{ w \in \Sigma^* \mid w \notin A \}$.

{\bf Proof idea}:


{\bf Proof}: 




\vfill

Application: Design a finite automaton that recognizes the language of all strings over $\{a,b\}$
whose length is not a multiple of $3$.

\vfill

{\bf Note}: On Friday, we'll see a new kind of finite automaton. It will be helpful to distinguish it from the
machines we've been talking about so we'll use {\bf Deterministic Finite Automaton} (DFA) to refer to the machines 
from Section 1.1. 

\newpage
\subsection*{Friday: Nondeterministic automata}




\begin{center}
\begin{tabular}{|ll|}
\hline
\multicolumn{2}{|l|}{{\bf Nondeterministic finite automaton}  (Sipser Page 53) Given as $M = (Q, \Sigma, \delta, q_0, F)$}\\
& \\
Finite set of states $Q$  & Can  be labelled by any collection  of distinct names. Default: $q0, q1, \ldots$  \\
Alphabet $\Sigma$ &  Each input to the automaton is a string over  $\Sigma$. \\
Arrow labels $\Sigma_\varepsilon$ &  $\Sigma_\varepsilon = \Sigma \cup \{ \varepsilon\}$. \\
&  Arrows 
in the state diagram are labelled either by symbols from $\Sigma$ or by $\varepsilon$ \\
Transition function $\delta$  & $\delta: Q \times \Sigma_{\varepsilon} \to \mathcal{P}(Q)$
gives the {\bf set of possible next states} for a transition \\
&  from the current state upon reading a symbol or spontaneously moving.\\
Start state $q_0$ & Element of $Q$.  Each computation of the machine starts at the  start  state.\\
Accept (final) states $F$ & $F \subseteq  Q$.\\
& \\
\multicolumn{2}{|p{\textwidth}|}{$M$ accepts the input string $w \in \Sigma^*$ if and only if {\bf there is} a computation of $M$ on 
$w$ that processes the whole string and ends in an
accept state.}\\
\hline
\end{tabular}
\end{center}

The formal definition of the NFA over $\{0,1\}$ given by this state diagram is: 

\includegraphics[width=2in]{../../resources/machines/Lect4NFA1.png}

The language over $\{0,1\}$ recognized by this NFA is:

\vspace{70pt}

Change the transition function to get a different NFA which accepts
the empty string (and potentially other strings too).


\newpage

The state diagram of an NFA over $\{a,b\}$ is below.  The formal definition of this NFA is:

\vspace{-30pt}

\includegraphics[width=2.5in]{../../resources/machines/Lect5NFA1.png}


\vspace{-10pt}

The language recognized by this NFA is:  


\newpage
\subsection*{Week 2 at a glance}

\subsubsection*{Textbook reading: Section 1.1, 1.2}

{\it For Wednesday}: Pages 41-43 (Figures 1.18, 1.19, 1.20) (examples of automata and languages).

{\it For Friday}: Pages 48-50 (Figures 1.27, 1.29) (introduction to nondeterminism).

{\it For Week 3 Monday}: Pages 60-61 Theorem 1.47 and Theorem 1.48 (closure proofs).


\subsubsection*{Make sure you can:}

\begin{itemize}
\item Use regular expressions and relate them to languages and automata
\begin{itemize}

   \item Write and debug regular expressions using correct syntax

\end{itemize}

\item Use precise notation to formally define the state diagram of DFA, NFA and 
use clear English to describe computations of DFA, NFA informally.

\begin{itemize}
   \item Design an automaton that recognizes a given language

   \item Specify a general construction for DFA based on parameters

   \item Design general constructions for DFA

   \item Motivate the use of nondeterminism

   \item Trace the computation(s) of a nondeterministic finite automaton

\end{itemize}






\end{itemize}

\subsubsection*{TODO:}
\begin{list}
   {\itemsep2pt}
   \item \#FinAid Assignment on Canvas https://canvas.ucsd.edu/courses/51649/quizzes/158899
   \item Review quizzes based on class material each day. 
   \item Homework assignment 1 due Thursday.
\end{list}

\end{document}