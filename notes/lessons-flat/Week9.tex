\documentclass[12pt, oneside]{article}

\usepackage[letterpaper, scale=0.89, centering]{geometry}
\usepackage{fancyhdr}
\setlength{\parindent}{0em}
\setlength{\parskip}{1em}

\usepackage{tikz}
\usetikzlibrary{automata,positioning,arrows}

\pagestyle{fancy}
\fancyhf{}
\renewcommand{\headrulewidth}{0pt}
\rfoot{\href{https://creativecommons.org/licenses/by-nc-sa/2.0/}{CC BY-NC-SA 2.0} Version \today~(\thepage)}

\usepackage{amssymb,amsmath,pifont,amsfonts,comment,enumerate,enumitem}
\usepackage{currfile,xstring,hyperref,tabularx,graphicx,wasysym}
\usepackage[labelformat=empty]{caption}
\usepackage{xcolor}
\usepackage{multicol,multirow,array,listings,tabularx,lastpage,textcomp,booktabs}

\lstnewenvironment{algorithm}[1][] {   
    \lstset{ mathescape=true,
        frame=tB,
        numbers=left, 
        numberstyle=\tiny,
        basicstyle=\rmfamily\scriptsize, 
        keywordstyle=\color{black}\bfseries,
        keywords={,procedure, div, for, to, input, output, return, datatype, function, in, if, else, foreach, while, begin, end, }
        numbers=left,
        xleftmargin=.04\textwidth,
        #1
    }
}
{}

\newcommand\abs[1]{\lvert~#1~\rvert}
\newcommand{\st}{\mid}

\newcommand{\cmark}{\ding{51}}
\newcommand{\xmark}{\ding{55}}
 
\begin{document}
\begin{flushright}
    \StrBefore{\currfilename}{.}
\end{flushright} 
\subsection*{Monday - Memorial Day}
No class today.

\subsection*{Wednesday}





Recall:  $A$ is  {\bf  mapping  reducible to} $B$, written $A \leq_m B$,  means there is a computable function 
$f : \Sigma^* \to \Sigma^*$ such that {\it for all} strings  $x$ in $\Sigma^*$, 
\[
x  \in  A \qquad \qquad \text{if and  only  if} \qquad \qquad f(x) \in B.
\]

True or False: $\overline{A_{TM}} \leq_m \overline{HALT_{TM}}$

\vfill

True or False: $HALT_{TM} \leq_m A_{TM}$.

\vfill


{\bf Theorem} (Sipser 5.28): If $A \leq_m B$ and $B$ is recognizable, then $A$ is recognizable.

{\bf Proof}: 

\vfill

{\bf Corollary}: If  $A \leq_m B$ and $A$ is unrecognizable, then $B$ is unrecognizable.

\vfill

\newpage
{\it Strategy}:  

(i) To prove that a  recognizable language $R$ is undecidable, prove that $A_{TM} \leq_m R$.


(ii) To prove that a co-recognizable language $U$ is undecidable, prove that $\overline{A_{TM}} \leq_m U$,
 i.e. that $A_{TM} \leq_m \overline{U}$.

 

\[
E_{TM} = \{ \langle M \rangle \mid \text{$M$ is a Turing machine and $L(M) = \emptyset$} \}
\]

Example  string in  $E_{TM}$ is \underline{\phantom{\hspace{1.6in}}} .
Example  string not  in  $E_{TM}$ is \underline{\phantom{\hspace{1.6in}}} .

$E_{TM}$ is ~~decidable /  undecidable~~ and ~~recognizable /  unrecognizable~~.

$\overline{E_{TM}}$ is ~~decidable /  undecidable~~ and ~~recognizable /  unrecognizable~~.


{\bf Claim}: $\underline{\phantom{\hspace{1.6in}}}  \leq_m \overline{E_{TM}}$.

{\bf Proof}: Need computable function  $F: \Sigma^* \to \Sigma^*$  such that  $x \in A_{TM}$ iff $F(x)  \notin  E_{TM}$.
Define
\begin{quote}
$F =  ``$ On input $x$,
\begin{itemize}
\item[1.] Type-check whether  $x = \langle M, w \rangle$ for some TM $M$ and string $w$.   If so, move to step 2; if  not, output 
\item[2.] Construct the following machine $M'_x$:
\vspace{50pt}
\item[3.] Output $\langle M'_x \rangle$."
\end{itemize}
\end{quote}

Verifying correctness:
\begin{center}
\begin{tabular}{|c|c|}
\hline
Input string &  Output string \\
\hline
$\langle M, w \rangle$ where  $w \in L(M)$ & \phantom{\hspace{4in}} \\
& \\
$\langle M, w \rangle$ where $w \notin L(M)$ & \\
&\\
$x$ not encoding any pair of  TM and string   &  \\
\hline
\end{tabular}
\end{center}

\newpage 
\subsection*{Review: Week 9 Wednesday}

Please complete the review quiz questions on \href{http://gradescope.com}{Gradescope} about 
mapping reductions.

{\bf Pre class reading for next time}: Introduction to Chapter 7.



\newpage
\subsection*{Friday}



Recall:  $A$ is  {\bf  mapping  reducible to} $B$, written $A \leq_m B$,  means there is a computable function 
$f : \Sigma^* \to \Sigma^*$ such that {\it for all} strings  $x$ in $\Sigma^*$, 
\[
x  \in  A \qquad \qquad \text{if and  only  if} \qquad \qquad f(x) \in B.
\]
\[
EQ_{TM} = \{ \langle M, M' \rangle \mid \text{$M$ and $M'$ are both Turing machines and $L(M) =L(M')$} \}
\]

Example  string in  $EQ_{TM}$ is \underline{\phantom{\hspace{1.5in}}} .
Example  string not  in  $EQ_{TM}$ is \underline{\phantom{\hspace{1.5in}}} .

$EQ_{TM}$ is ~~decidable /  undecidable~~ and ~~recognizable /  unrecognizable~~.

$\overline{EQ_{TM}}$ is ~~decidable /  undecidable~~ and ~~recognizable /  unrecognizable~~.

To  prove, show that  $\underline{\phantom{\hspace{1.6in}}}  \leq_m EQ_{TM}$
and that $\underline{\phantom{\hspace{1.6in}}}  \leq_m \overline{EQ_{TM}}$.


\vfill

Verifying correctness:
\begin{center}
\begin{tabular}{|c|c|}
\hline
Input string &  Output string \\
\hline
$\langle M, w \rangle$ where  $M$ halts on $w$ & \phantom{\hspace{4in}} \\
& \\
$\langle M, w \rangle$ where $M$ loops on $w$ & \\
&\\
$x$ not encoding any pair of  TM and string   &  \\
\hline
\end{tabular}
\end{center}

\vfill

\newpage

In practice, computers (and Turing machines) don't have infinite tape, 
and we can't afford to wait unboundedly long for an answer.
``Decidable" isn't good enough - we want ``Efficiently decidable".

For a given algorithm working on a given input, how long do we need to wait for an answer? 
How does the running time depend on the input in the worst-case? average-case? 
We expect to have to spend more time on computations with larger inputs.


A language is {\bf recognizable} if \underline{\phantom{\hspace{4.5in}}}

A language is {\bf decidable} if \underline{\phantom{\hspace{4.7in}}}

A language is {\bf efficiently  decidable} if \underline{\phantom{\hspace{4in}}}

A function is {\bf computable} if \underline{\phantom{\hspace{4.7in}}}

A function is {\bf efficiently computable} if \underline{\phantom{\hspace{4in}}}\\

\vfill

Definition  (Sipser 7.1): For  $M$ a deterministic decider, its {\bf running time} is the function  $f: \mathbb{N} \to \mathbb{N}$
given  by
\[
f(n) =  \text{max number of  steps $M$ takes before halting, over all inputs  of length $n$}
\]

Definition (Sipser 7.7): For each function $t(n)$, the {\bf time complexity class}  $TIME(t(n))$, is defined  by
\[
TIME( t(n)) = \{ L \mid \text{$L$ is decidable by  a Turing machine with running time in  $O(t(n))$} \}
\]

An example of an element of  $TIME(  1  )$ is 

An example of an element of  $TIME(  n  )$ is 


Note: $TIME( 1) \subseteq TIME (n)  \subseteq TIME(n^2)$

\vfill

Definition (Sipser 7.12) : $P$ is the class of languages that  are decidable in polynomial time on 
a deterministic 1-tape  Turing  machine
\[
P  =  \bigcup_k TIME(n^k)
\]

{\it Compare to exponential time: brute-force search.}


Theorem (Sipser 7.8): Let $t(n)$ be a  function with  $t(n)  \geq n$.  Then every $t(n)$ time deterministic 
multitape Turing machine has an equivalent $O(t^2(n))$ time deterministic 1-tape Turing machine.

\newpage 
\newpage
\subsection*{Review: Week 9 Friday}


Please complete the review quiz questions on \href{http://gradescope.com}{Gradescope} about 
complexity.

{\bf Pre class reading for next time}: Skim Chapter 7.

\end{document}